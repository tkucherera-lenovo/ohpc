% -*- mode: latex; fill-column: 120; -*-

With the provisioning services enabled, the next step is to define
a system image that can subsequently be
used to provision one or more {\em compute} nodes. The following subsections highlight this process.

\subsubsection{Build initial BOS image} \label{sec:assemble_bos}
The following steps illustrate the process to build a minimal, default image for use with \Confluent{}. To begin, you will
first need to have a local copy of the ISO image available for the underlying OS. In this recipe, the relevant ISO image
is \texttt{Rocky-9.4-x86\_64-dvd1.iso} (available from the Rocky
\href{https://rockylinux.org/download/}{\color{blue}download} page).
We initialize the image
creation process using the \texttt{osdeploy} command assuming that the necessary ISO image is available locally in
\texttt{\$\{iso\_path\}} as follows:

The \texttt{osdeploy initialize} command is used to prepare a confluent server to deploy deploy operating systems.
For first time setup, run osdeploy initialize interactively to be walked through the various options using: 
\texttt{osdeploy initialize -i}

% begin_ohpc_run
% ohpc_comment_header Initialize OS images for use with Confluent \ref{sec:assemble_bos}
\begin{lstlisting}[language=bash,literate={-}{-}1,keywords={},upquote=true,keepspaces,literate={BOSVER}{\baseos{}}1]
[sms](*\#*) osdeploy initialize -${initialize_options}
[sms](*\#*) osdeploy import ${iso_path}

\end{lstlisting}
% end_ohpc_run

\noindent Once completed, OS image should be available for use within \Confluent{}. These can be queried via:

\begin{lstlisting}[language=bash,literate={-}{-}1,keywords={},upquote=true,keepspaces,literate={BOSVER}{\baseos{}}1]
# Query available images
[sms](*\#*) osdeploy list
Distributions:
  rocky-8.5-x86_64
  rocky-9.4-x86_64
Profiles:
  rhel-9.4-x86_64-default
  rocky-8.5-x86_64-default
\end{lstlisting}

If needing to copy files from the sms node to the compute nodes during deployment, this can be done by
modifying the syncfiles file that is created when \texttt{osdeploy import} command is run. For an environment
that has no DNS server and needs to have /etc/hosts file synced amongst all the nodes, the following command
should be run.   

% begin_ohpc_run
% ohpc_validation_newline
% ohpc_validation_comment Sync the hosts file in cluster
\begin{lstlisting}[language=bash,literate={-}{-}1,keywords={},upquote=true,keepspaces,literate={BOSVER}{\baseos{}}1]
  [sms](*\#*) echo "/etc/hosts -> /etc/hosts" >> /var/lib/confluent/public/os/rocky-9.4-x86_64-default/syncfiles

\end{lstlisting}
% end_ohpc_run
  

%The \texttt{CHROOT} environment variable highlights the path and is used by
%subsequent commands to augment the basic installation.

