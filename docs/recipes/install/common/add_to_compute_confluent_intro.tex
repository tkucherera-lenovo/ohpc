% -*- mode: latex; fill-column: 120; -*-

The next step is adding \OHPC{} components to the {\em compute} nodes that at this
point are running basic OSes.  This process will leverage two \Confluent{}-provided
commands: \texttt{nodeshell} to run \texttt{\pkgmgr{}} installer on all the
nodes in parallel  and \texttt{nodersync} to distribute configuration files from the
SMS to the {\em compute} nodes.

\noindent To do this, repositories on the {\em compute} nodes need to be configured
properly.

\Confluent{} has automatically setup an  OS repository on the SMS and configured the
nodes to use it, but it has  also enabled online OS repositories.


\noindent Next, we alse add the OHPC repo to the compute nodes \S\ref{sec:enable_repo}

% begin_ohpc_run
% ohpc_comment_header Setup nodes repositories and Install OHPC components \ref{sec:add_components}
\begin{lstlisting}[language=bash,literate={-}{-}1,keywords={},upquote=true]
# Add OpenHPC repo 
[sms](*\#*) (*\chrootinstall*) http://repos.openhpc.community/OpenHPC/3/EL_9/x86_64/ohpc-release-3-1.el9.x86_64.rpm
\end{lstlisting}
% end_ohpc_run

The {\em compute} nodes also need access to the EPEL repository, a required
dependency for \OHPC{} packages. 

% begin_ohpc_run
% ohpc_comment_header Configure access to EPEL repo
\begin{lstlisting}[language=bash,literate={-}{-}1,keywords={},upquote=true]
# Add epel repo
[sms](*\#*) (*\chrootinstall*) epel-release

\end{lstlisting}
% end_ohpc_run


\noindent Additionally, a workaround is needed for \OHPC{} documentation files,
which are installed into a read-only NFS share /opt/ohpc/pub. Any package
attempting to write to that directory will fail to install. The following
prevents that by directing \texttt{rpm} not to install documentation files on
the {\em compute} nodes:

% begin_ohpc_run
\begin{lstlisting}[language=bash,literate={-}{-}1,keywords={},upquote=true]
[sms](*\#*) nodeshell compute echo -e %_excludedocs 1 \>\> ~/.rpmmacros
\end{lstlisting}
% end_ohpc_run

\noindent Now \OHPC{} and other cluster-related software components can be
installed on the nodes. The first step is to install a base compute package:
% begin_ohpc_run
% ohpc_comment_header Add OpenHPC base components to compute image
\begin{lstlisting}[language=bash,literate={-}{-}1,keywords={},upquote=true]
# Install compute node base meta-package
[sms](*\#*) (*\chrootinstall*) ohpc-base-compute
\end{lstlisting}
% end_ohpc_run

\noindent Next, we can include additional components:
