% -*- mode: latex; fill-column: 120; -*-

The next step is adding \OHPC{} components to the {\em compute} nodes that at this
point are running basic OSes.  This process will leverage two \Confluent{}-provided
commands: \texttt{nodeshell} to run \texttt{\pkgmgr{}} installer on all the
nodes in parallel  and \texttt{nodersync} to distribute configuration files from the
SMS to the {\em compute} nodes.

\noindent To do this, repositories on the {\em compute} nodes need to be configured
properly.

\Confluent{} has automatically setup an  OS repository on the SMS and configured the
nodes to use it, but it has  also enabled online OS repositories, which will
cause  \texttt{\pkgmgr{}} failures if there is no connection to the public internet.
To avoid this, we disable the online repositories:

% begin_ohpc_run
% ohpc_comment_header Setup nodes repositories and Install OHPC components \ref{sec:add_components}
\begin{lstlisting}[language=bash,literate={-}{-}1,keywords={},upquote=true]
# Install dnf-plugins-core to have dnf config-manager available
[sms](*\#*) nodeshell compute dnf --setopt=\*.skip_if_unavailable=1 -y install dnf-plugins-core perl

# Enable CentOS-PowerTools repository
[sms](*\#*) nodeshell compute dnf config-manager --enable crb
\end{lstlisting}
% end_ohpc_run

\noindent Next, the nodes are configured to use the local \OHPC{} repository
setup in on the SMS in \S\ref{sec:enable_repo}

% begin_ohpc_run
\begin{lstlisting}[language=bash,literate={-}{-}1,keywords={},upquote=true]
# Add OpenHPC repo mirror hosted on SMS
[sms](*\#*) nodeshell compute dnf config-manager --add-repo=http://$sms_ip/$ohpc_repo_remote_dir/OpenHPC.local.repo
# Replace local path with SMS URL
[sms](*\#*) nodeshell compute "perl -pi -e 's/file:\/\/\@PATH\@/http:\/\/$sms_ip\/"${ohpc_repo_remote_dir//\//"\/"}"/s' \
            /etc/yum.repos.d/OpenHPC.local.repo"
\end{lstlisting}
% end_ohpc_run

The {\em compute} nodes also need access to the EPEL repository, a required
dependency for \OHPC{} packages. A straightforward way to accomplish this is to
host a local mirror on the {\em master} host. However, the repository is very
large ($>$10 GB) containing thousands of packages, while only 3 (\texttt{fping,
libconfuse, and libunwind}) are required for packages installed this guide.  Therefore,
here we only download the required packages and put them into a repository
accessible to the nodes.  Users wanting to mirror the entire EPEL repository
instead can take advantage of the \texttt{reposync} utility and should expect a
download time of over an hour depending on their network connection and a choice
of EPEL mirror.

% begin_ohpc_run
% ohpc_comment_header Configure access to EPEL packages
\begin{lstlisting}[language=bash,literate={-}{-}1,keywords={},upquote=true]
# Create directory holding EPEL packages
[sms](*\#*) mkdir -p $epel_repo_dir_confluent
[sms](*\#*) (*\install*) dnf-plugins-core createrepo
# Download required EPEL packages
[sms](*\#*) dnf download --destdir $epel_repo_dir_confluent fping libconfuse libunwind
# Create a repository from the downloaded packages
[sms](*\#*) createrepo $epel_repo_dir_confluent

# Setup access to the repo from the compute hosts
[sms](*\#*) nodeshell compute dnf config-manager --add-repo=http://$sms_ip/$epel_repo_remote_dir
[sms](*\#*) nodeshell compute "echo gpgcheck=0 >> /etc/yum.repos.d/${sms_ip}_"${epel_repo_remote_dir//\//"_"}.repo
\end{lstlisting}
% end_ohpc_run


\noindent Additionally, a workaround is needed for \OHPC{} documentation files,
which are installed into a read-only NFS share /opt/ohpc/pub. Any package
attempting to write to that directory will fail to install. The following
prevents that by directing \texttt{rpm} not to install documentation files on
the {\em compute} nodes:

% begin_ohpc_run
\begin{lstlisting}[language=bash,literate={-}{-}1,keywords={},upquote=true]
[sms](*\#*) nodeshell compute echo -e %_excludedocs 1 \>\> ~/.rpmmacros
\end{lstlisting}
% end_ohpc_run

\noindent Now \OHPC{} and other cluster-related software components can be
installed on the nodes. The first step is to install a base compute package:
% begin_ohpc_run
% ohpc_comment_header Add OpenHPC base components to compute image
\begin{lstlisting}[language=bash,literate={-}{-}1,keywords={},upquote=true]
# Install compute node base meta-package
[sms](*\#*) (*\chrootinstall*) ohpc-base-compute
\end{lstlisting}
% end_ohpc_run

\noindent Next, we can include additional components:
