\documentclass[letterpaper]{article}
\usepackage{common/ohpc-doc}
\setcounter{secnumdepth}{5}
\setcounter{tocdepth}{5}

% Include git variables
\input{vc.tex}

% Define Base OS and other local macros
\newcommand{\baseOS}{CentOS8.4}
\newcommand{\OSRepo}{CentOS\_8.4}
\newcommand{\OSTree}{CentOS\_8}
\newcommand{\OSTag}{el8}
\newcommand{\baseos}{centos8.4}
\newcommand{\baseosshort}{centos8}
\newcommand{\provisioner}{xCAT}
\newcommand{\provheader}{xCAT (stateful)}
\newcommand{\rms}{SLURM}
\newcommand{\rmsshort}{slurm}
\newcommand{\arch}{x86\_64}
\newcommand{\installimage}{install}
%%% WARNING: Hack below. The version should be read from ohpc-doc.sty, but the
%%% perl parsing script does not read that file. This works for one release, but
%%% needs a proper fix.
\newcommand{\VERLONG}{2.0}

% Define package manager commands
\newcommand{\pkgmgr}{dnf}
\newcommand{\addrepo}{wget -P /etc/yum.repos.d}
\newcommand{\chrootaddrepo}{wget -P \$CHROOT/etc/yum.repos.d}
\newcommand{\clean}{dnf clean expire-cache}
\newcommand{\chrootclean}{dnf --installroot=\$CHROOT clean expire-cache}
\newcommand{\install}{dnf -y install}
\newcommand{\chrootinstall}{psh compute dnf -y install}
\newcommand{\groupinstall}{dnf -y groupinstall}
\newcommand{\groupchrootinstall}{psh compute dnf -y groupinstall}
\newcommand{\remove}{dnf -y remove}
\newcommand{\upgrade}{dnf -y upgrade}
\newcommand{\chrootupgrade}{dnf -y --installroot=\$CHROOT upgrade}
\newcommand{\tftppkg}{syslinux-tftpboot}
\newcommand{\beegfsrepo}{https://www.beegfs.io/release/beegfs\_7.2.1/dists/beegfs-rhel8.repo}

% boolean for os-specific formatting
\toggletrue{isCentOS}
\toggletrue{isCentOS_ww_slurm_x86}
\toggletrue{isSLURM}
\toggletrue{isx86}
\toggletrue{isxCAT}
\toggletrue{isxCATstateful}
\toggletrue{isCentOS_x86}

\begin{document}
\graphicspath{{common/figures/}}
\thispagestyle{empty}

% Title Page
\input{common/title}
% Disclaimer
\input{common/legal}

\newpage
\tableofcontents
\newpage

% Introduction  --------------------------------------------------

\section{Introduction} \label{sec:introduction}
\input{common/install_header}
\input{common/intro} \\

\input{common/base_edition/edition}
\input{common/audience}
\input{common/requirements}
\input{common/inputs}


% Base Operating System --------------------------------------------

\vspace*{0.2cm}
\section{Install Base Operating System (BOS)}
\input{common/bos}

%\clearpage
% begin_ohpc_run
% ohpc_validation_newline
% ohpc_validation_comment Disable firewall
\begin{lstlisting}[language=bash,keywords={}]
[sms](*\#*) systemctl disable firewalld
[sms](*\#*) systemctl stop firewalld
\end{lstlisting}
% end_ohpc_run

% ------------------------------------------------------------------

\section{Install \xCAT{} and Provision Nodes with BOS} \label{sec:provision_compute_bos}
\input{common/xcat_stateful_compute_bos_intro}

\subsection{Enable \xCAT{} repository for local use} \label{sec:enable_xcat}
\input{common/enable_xcat_repo}

\noindent \xCAT{} has a number of dependencies that are required for
installation that are housed in separate public repositories for various
distributions. To enable for local use, issue the following:

% begin_ohpc_run
\begin{lstlisting}[language=bash,keywords={},basicstyle=\fontencoding{T1}\fontsize{8.0}{10}\ttfamily,literate={ARCH}{\arch{}}1 {-}{-}1]
[sms](*\#*) (*\install*) centos-release-stream
[sms](*\#*) (*\addrepo*) https://xcat.org/files/xcat/repos/yum/xcat-dep/rh8/ARCH/xcat-dep.repo
\end{lstlisting}
% end_ohpc_run

\subsection{Add provisioning services on {\em master} node} \label{sec:add_provisioning}
\input{common/install_provisioning_xcat_intro_stateful}
%\input{common/enable_pxe}

\vspace*{-0.15cm}
\subsection{Complete basic \xCAT{} setup for {\em master} node} \label{sec:setup_xcat}
\input{common/xcat_setup}


\subsection{Define {\em compute} image for provisioning}
\input{common/xcat_init_os_images_centos}

\clearpage
\subsection{Add compute nodes into \xCAT{} database} \label{sec:xcat_add_nodes}
\input{common/add_xcat_hosts_intro}

%\vspace*{-0.25cm}
\subsection{Boot compute nodes} \label{sec:boot_computes}
\input{common/reset_computes_xcat}

\section{Install \OHPC{} Components} \label{sec:basic_install}
\input{common/install_ohpc_components_intro}


\subsection{Enable \OHPC{} repository for local use} \label{sec:enable_repo}
\OHPC{} provides an online package repository, but here we setup and use a local
mirror on the SMS host for \OHPC{} package installation. This is preferable for
a stateful install, because {\em compute} nodes can access it on the local
network.  


Assuming that the {\em master} has outside network access, the mirror is
created  by downloading a tarball of the repository, unpacking it and running a
provided script to configure the package manager.

% begin_ohpc_run
% ohpc_validation_newline
% ohpc_comment_header Download OHPC repo and create local mirror \ref{sec:enable_repo}
\begin{lstlisting}[language=bash,keywords={},basicstyle=\fontencoding{T1}\fontsize{7.6}{10}\ttfamily,
	literate={VER}{\OHPCVerTree{}}1 {OSTREE}{\OSTree{}}1 {TAG}{\OSTag{}}1 {ARCH}{\arch{}}1 {-}{-}1 
        {VERLONG}{\OHPCVersion{}}1]
# Download OHPC tarball for local installation
[sms](*\#*) wget http://repos.openhpc.community/dist/VERLONG/OpenHPC-VERLONG.OSTREE.ARCH.tar
# Create directory if necessary
[sms](*\#*) mkdir -p $ohpc_repo_dir
#Unpack
[sms](*\#*) tar xvf OpenHPC-VERLONG.OSTREE.ARCH.tar -C $ohpc_repo_dir
#Create local mirror repository
[sms](*\#*) $ohpc_repo_dir/make_repo.sh
\end{lstlisting}
% end_ohpc_run


% begin_ohpc_run
% ohpc_validation_newline
% ohpc_validation_comment Verify OpenHPC repository has been enabled before proceeding
% ohpc_validation_newline
% ohpc_command dnf repolist | grep -q OpenHPC
% ohpc_command if [ $? -ne 0 ];then
% ohpc_command    echo "Error: OpenHPC repository must be enabled locally"
% ohpc_command    exit 1
% ohpc_command fi
% end_ohpc_run


In addition to the \OHPC{} and \xCAT{} package repositories, the {\em master} host also
requires access to the standard base OS distro repositories in order to resolve
necessary dependencies. For \baseOS{}, the requirements are to have access to
both the base OS and EPEL repositories for which mirrors are freely available online:

\begin{itemize*}
\item CentOS-8 - Base 8.3.2011
  (e.g. \href{http://mirror.centos.org/centos-8/8/BaseOS/x86\_64/os}
             {\color{blue}{http://mirror.centos.org/centos-8/8/BaseOS/x86\_64/os}} )
\item EPEL 8 (e.g. \href{http://download.fedoraproject.org/pub/epel/8/Everything/x86\_64}
                        {\color{blue}{http://download.fedoraproject.org/pub/epel/8/Everything/x86\_64}} )
\end{itemize*}

\noindent The public EPEL repository is enabled by installing
\texttt{epel-release} package. Note that this requires the CentOS Extras
repository, which is shipped with CentOS and is enabled by default.

% begin_ohpc_run
\begin{lstlisting}[language=bash,keywords={},basicstyle=\fontencoding{T1}\fontsize{8.0}{10}\ttfamily,literate={ARCH}{\arch{}}1 {-}{-}1]
[sms](*\#*)  (*\install*) epel-release
\end{lstlisting}
% end_ohpc_run

Now \OHPC{} packages can be installed. To add the base package on the SMS
issue the following
% begin_ohpc_run
\begin{lstlisting}[language=bash,keywords={},basicstyle=\fontencoding{T1}\fontsize{8.0}{10}\ttfamily,literate={ARCH}{\arch{}}1 {-}{-}1]
[sms](*\#*)  (*\install*) ohpc-base
\end{lstlisting}
% end_ohpc_run


\input{common/automation}


\subsection{Setup time synchronization service on {\em master} node} \label{sec:add_ntp}
\input{common/time}

\subsection{Add resource management services on {\em master} node} \label{sec:add_rm}
\input{common/install_slurm}

\subsection{Optionally add \InfiniBand{} support services on {\em master} node} \label{sec:add_ofed}
\input{common/ibsupport_sms_centos}
\vspace*{0.3cm}
\subsection{Optionally add \OmniPath{} support services on {\em master} node} \label{sec:add_opa}
\input{common/opasupport_sms_centos}

%\vspace*{0.5cm}
\clearpage
\subsubsection{Add \OHPC{} components to {\em compute} nodes} \label{sec:add_components}
\input{common/add_to_compute_stateful_xcat_intro}

%\newpage
% begin_ohpc_run
% ohpc_validation_comment Add OpenHPC components to compute instance
\begin{lstlisting}[language=bash,literate={-}{-}1,keywords={},upquote=true]
# Add Slurm client support meta-package
[sms](*\#*) (*\chrootinstall*) ohpc-slurm-client

# Add Network Time Protocol (NTP) support
[sms](*\#*) (*\chrootinstall*) ntp

# Add kernel drivers
[sms](*\#*) (*\chrootinstall*) kernel

# Include modules user environment
[sms](*\#*) (*\chrootinstall*)  --enablerepo=powertools lmod-ohpc
\end{lstlisting}
% end_ohpc_run

% ohpc_comment_header Optionally add InfiniBand support services in compute node image \ref{sec:add_components}
% ohpc_command if [[ ${enable_ib} -eq 1 ]];then
% ohpc_indent 5
\begin{lstlisting}[language=bash,literate={-}{-}1,keywords={},upquote=true]
# Optionally add IB support and enable
[sms](*\#*) (*\groupchrootinstall*) "InfiniBand Support"
\end{lstlisting}
% ohpc_indent 0
% ohpc_command fi
% end_ohpc_run

\vspace*{-0.25cm}
\subsubsection{Customize system configuration} \label{sec:master_customization}
\input{common/xcat_stateful_customize_centos}

% Additional commands when additional computes are requested

% begin_ohpc_run
% ohpc_validation_newline
% ohpc_validation_comment Update basic slurm configuration if additional computes defined
% ohpc_validation_comment This is performed on the SMS, nodes will pick it up config file is copied there later
% ohpc_command if [ ${num_computes} -gt 4 ];then
% ohpc_command    perl -pi -e "s/^NodeName=(\S+)/NodeName=${compute_prefix}[1-${num_computes}]/" /etc/slurm/slurm.conf
% ohpc_command    perl -pi -e "s/^PartitionName=normal Nodes=(\S+)/PartitionName=normal Nodes=${compute_prefix}[1-${num_computes}]/" /etc/slurm/slurm.conf
% ohpc_command fi
% end_ohpc_run

%\clearpage
\subsubsection{Additional Customization ({\em optional})} \label{sec:addl_customizations}
\input{common/compute_customizations_intro}

\paragraph{Increase locked memory limits}
\input{common/memlimits_stateful}

\paragraph{Enable ssh control via resource manager}
\input{common/slurm_pam_stateful}

\paragraph{Add \Lustre{} client} \label{sec:lustre_client}
\input{common/lustre-client}
\input{common/lustre-client-centos-stateful}
\input{common/lustre-client-post-stateful}

\vspace*{0.4cm}

\paragraph{Add \clustershell{}}
\input{common/clustershell}

\paragraph{Add \genders{}}
\input{common/genders}

\paragraph{Add Magpie}
\input{common/magpie}

\paragraph{Add \conman{}} \label{sec:add_conman}
\input{common/conman}

\paragraph{Add \nhc{}} \label{sec:add_nhc}
\input{common/nhc}
\input{common/nhc_slurm}

%\subsubsection{Identify files for synchronization} \label{sec:file_import}
%\input{common/import_xcat_files}
%\input{common/import_xcat_files_slurm}

%%%\subsubsection{Optional kernel arguments} \label{sec:optional_kargs}
%%%\input{common/conman_post}

\section{Install \OHPC{} Development Components}
\input{common/dev_intro.tex}

%\vspace*{-0.15cm}
%\clearpage
\subsection{Development Tools} \label{sec:install_dev_tools}
\input{common/dev_tools}

\vspace*{-0.15cm}
\subsection{Compilers} \label{sec:install_compilers}
\OHPC{} presently packages the \GNU{} compiler toolchain integrated with the
underlying Lmod modules system in a hierarchical fashion. The modules
system will conditionally present compiler-dependent software based on the
toolchain currently loaded.

% begin_ohpc_run
% ohpc_comment_header Install Compilers \ref{sec:install_compilers}
\begin{lstlisting}[language=bash]
[sms](*\#*) (*\install*) gnu14-compilers-ohpc
\end{lstlisting}
% end_ohpc_run

%%% The llvm compiler toolchains are also provided as a standalone additional
%%% compiler family (ie. users can easily switch between gcc/clang environments),
%%% but we do not provide the full complement of downstream library builds.
%%%
%%% % begin_ohpc_run
%%% % ohpc_comment_header Install llvm Compilers
%%% \begin{lstlisting}[language=bash]
%%% [sms](*\#*) (*\install*) llvm5-compilers-ohpc
%%% \end{lstlisting}
%%% % end_ohpc_run


%\clearpage
\subsection{MPI Stacks} \label{sec:mpi}
For MPI development and runtime support, \OHPC{} provides pre-packaged builds
for a variety of MPI families and transport layers. Currently available options
and their applicability to various network transports are summarized in
Table~\ref{table:mpi}.  The command that follows installs a starting set of MPI
families that are compatible with both ethernet and high-speed fabrics.

\iftoggleverb{isx86}
% x86_64

\begin{table}[h]
\caption{Available MPI variants} \label{table:mpi}
\centering
\begin{tabular}{@{\hspace*{0.2cm}} *5l @{}}    \toprule
                                  & Ethernet (TCP)                 & \InfiniBand{}                  & \IntelR{} Omni-Path            \\ \midrule
\rowcolor{black!10} MPICH (ofi) & \multicolumn{1}{c}{\checkmark} & \multicolumn{1}{c}{\checkmark} & \multicolumn{1}{c}{\checkmark} \\
 MPICH (ucx)       & \multicolumn{1}{c}{\checkmark} & \multicolumn{1}{c}{\checkmark} & \multicolumn{1}{c}{\checkmark} \\
\rowcolor{black!10} MVAPICH2                          &                                & \multicolumn{1}{c}{\checkmark} &                                \\
MVAPICH2 (psm2) &                              &                                & \multicolumn{1}{c}{\checkmark} \\
\rowcolor{black!10} OpenMPI (ofi/ucx)            & \multicolumn{1}{c}{\checkmark} & \multicolumn{1}{c}{\checkmark} & \multicolumn{1}{c}{\checkmark} \\
%\rowcolor{black!10} OpenMPI (PMIx) & \multicolumn{1}{c}{\checkmark} & \multicolumn{1}{c}{\checkmark} & \multicolumn{1}{c}{\checkmark} \\ \bottomrule
\end{tabular}
\end{table}

\else
% aarch64

\begin{table}[h]
\caption{Available MPI builds} \label{table:mpi}
\centering
\begin{tabular}{@{\hspace*{0.2cm}} *5l @{}}    \toprule
                                  & Ethernet (TCP)                 & \InfiniBand{}                              \\ \midrule
\rowcolor{black!10} MPICH         & \multicolumn{1}{c}{\checkmark} &                                            \\
\rowcolor{black!10} OpenMPI                           & \multicolumn{1}{c}{\checkmark} & \multicolumn{1}{c}{\checkmark} \\
\end{tabular}
\end{table}

\fi

% begin_ohpc_run
% ohpc_comment_header Install MPI Stacks \ref{sec:mpi}
% ohpc_command if [[ ${enable_mpi_defaults} -eq 1 ]];then
% ohpc_indent 5
\begin{lstlisting}[language=bash]
[sms](*\#*) (*\install*) openmpi5-pmix-gnu14-ohpc mpich-ofi-gnu14-ohpc
\end{lstlisting}
% ohpc_indent 0
% ohpc_command fi
% end_ohpc_run

Note that OpenHPC 2.x introduces the use of two related transport layers for
the MPICH and OpenMPI builds that support a variety of underlying
fabrics: \href{https://www.openucx.org}{UCX} (Unified Communication X)
and \href{https://ofiwg.github.io/libfabric/}{OFI} (OpenFabrics interfaces).
In the case of OpenMPI, a monolithic build is provided which supports both
transports and end-users can customize their runtime preferences with
environment variables. For MPICH, two separate builds are provided and the
example above highlighted installing the {\texttt ofi} variant.  However, the
packaging is designed such that both versions can be installed simultaneously
and users can switch between the two via normal module command
semantics. Alternatively, a site can choose to install the {\texttt ucx} variant
instead as a drop-in MPICH replacement:

\begin{lstlisting}[language=bash]
[sms](*\#*) (*\install*) mpich-ucx-gnu14-ohpc
\end{lstlisting}

In the case where both MPICH variants are installed, two modules will be
visible in the end-user environment and an example of this configuration is
highlighted is below.

\begin{lstlisting}[language=bash]
[sms](*\#*) module avail mpich

-------------------- /opt/ohpc/pub/moduledeps/gnu14---------------------
   mpich/3.4.3-ofi    mpich/3.4.3-ucx (D)
\end{lstlisting}

If your system includes \InfiniBand{} and you enabled underlying support in
\S\ref{sec:add_ofed} and \S\ref{sec:addl_customizations}, an additional
MVAPICH2 family is available for use:

% begin_ohpc_run
% ohpc_validation_newline
% ohpc_command if [[ ${enable_ib} -eq 1 ]];then
% ohpc_indent 5
\begin{lstlisting}[language=bash]
[sms](*\#*) (*\install*) mvapich2-gnu14-ohpc
\end{lstlisting}
% ohpc_indent 0
% ohpc_command fi
% end_ohpc_run

Alternatively, if your system includes \IntelR{} \OmniPath{}, use the (\texttt{psm2})
variant of MVAPICH2 instead:

% begin_ohpc_run
% ohpc_command if [[ ${enable_opa} -eq 1 ]];then
% ohpc_indent 5
\begin{lstlisting}[language=bash]
[sms](*\#*) (*\install*) mvapich2-psm2-gnu14-ohpc
\end{lstlisting}
% ohpc_indent 0
% ohpc_command fi
% end_ohpc_run

%%--
%% https://github.com/openhpc/ohpc/issues/1273
%% disabling until we can get pmix/openmpi/slurm to play nicely
%%--
%% An additional OpenMPI build variant is listed in Table~\ref{table:mpi} which
%% enables \href{https://pmix.github.io/pmix/}{\color{blue}{PMIx}} job launch
%% support for use with \SLURM{}. This optional variant is
%% available as \texttt{openmpi5-pmix-slurm-gnu9-ohpc}.


\subsection{Performance Tools} \label{sec:install_perf_tools}
\OHPC{} provides a variety of open-source tools to aid in application
performance analysis (refer to Appendix~\ref{appendix:manifest} for a listing
of available packages). This group of tools can be installed as follows:

% begin_ohpc_run
% ohpc_comment_header Install Performance Tools \ref{sec:install_perf_tools}
\begin{lstlisting}[language=bash,keywords={},literate={-}{-}1]
# Install perf-tools meta-package
[sms](*\#*) (*\install*) ohpc-gnu14-perf-tools
\end{lstlisting}
% end_ohpc_run


\subsection{Setup default development environment}
System users often find it convenient to have a default development environment
in place so that compilation can be performed directly for parallel programs
requiring MPI. This setup can be conveniently enabled via modules and the \OHPC{}
modules environment is pre-configured to load an \texttt{ohpc} module on login
(if present). The following package install provides a default
environment that enables autotools, the \GNU{} compiler toolchain, and the
OpenMPI stack.

% begin_ohpc_run
\begin{lstlisting}[language=bash]
[sms](*\#*) (*\install*) lmod-defaults-gnu14-openmpi5-ohpc
\end{lstlisting}
% end_ohpc_run

\begin{center}
\begin{tcolorbox}[]
\small
\iftoggleverb{isx86}
If you want to change the default environment from the suggestion above, \OHPC{}
also provides the \GNU{} compiler toolchain with the MPICH and MVAPICH2 stacks:
\fi

\iftoggleverb{isaarch}
If you want to change the default environment from the suggestion above, \OHPC{}
also provides additional default options using the \GNU{} compiler toolchain
with multiple MPICH variants or MVAPICH2. Relevant lmod-default packages names
are as follows:
\fi

\begin{itemize*}
\item lmod-defaults-gnu14-mpich-ofi-ohpc
\item lmod-defaults-gnu14-mpich-ucx-ohpc
\iftoggleverb{isx86}
\item lmod-defaults-gnu14-mvapich2-ohpc
\fi
\end{itemize*}
\end{tcolorbox}
\end{center}


%\vspace*{0.2cm}
\subsection{3rd Party Libraries and Tools} \label{sec:3rdparty}
\input{common/third_party_libs_intro}

\begin{center}
\begin{tcolorbox}[]
\small
\OHPC{}-provided 3rd party builds are configured to be installed
into a common top-level repository so that they can be easily exported to
desired hosts within the cluster. This common top-level path
(\path{/opt/ohpc/pub}) was previously configured to be mounted on {\em
 compute} nodes in \S\ref{sec:master_customization}, so the packages will be
immediately available for use on the cluster after installation on the {\em
 master} host.
\end{tcolorbox}
\end{center}

%\iftoggle{isCentOS}{\clearpage}
%\nottoggle{isCentOS}{\clearpage}

For convenience, \OHPC{} provides package aliases for these 3rd party libraries
and utilities that can be used to install available libraries for use with the
GNU compiler family toolchain. For parallel libraries, aliases are grouped by
MPI family toolchain so that administrators can choose a subset should they
favor a particular MPI stack.  Please refer to Appendix~\ref{appendix:manifest}
for a more detailed listing of all available packages in each of these functional
areas. To install all available package offerings within \OHPC{}, issue the
following:

% begin_ohpc_run
% ohpc_comment_header Install 3rd Party Libraries and Tools \ref{sec:3rdparty}
\begin{lstlisting}[language=bash,keywords={},upquote=true,keepspaces]
# Install 3rd party libraries/tools meta-packages built with GNU toolchain
[sms](*\#*) (*\install*) ohpc-gnu14-serial-libs
[sms](*\#*) (*\install*) ohpc-gnu14-io-libs
[sms](*\#*) (*\install*) ohpc-gnu14-python-libs
[sms](*\#*) (*\install*) ohpc-gnu14-runtimes
\end{lstlisting}
% end_ohpc_run





% begin_ohpc_run
% ohpc_command if [[ ${enable_mpi_defaults} -eq 1 ]];then
% ohpc_indent 5
\begin{lstlisting}[language=bash,keywords={},upquote=true,keepspaces]
# Install parallel lib meta-packages for all available MPI toolchains
[sms](*\#*) (*\install*) ohpc-gnu14-mpich-parallel-libs
[sms](*\#*) (*\install*) ohpc-gnu14-openmpi5-parallel-libs
\end{lstlisting}
% ohpc_indent 0
% ohpc_command fi
% ohpc_command if [[ ${enable_ib} -eq 1 ]];then
% ohpc_indent 5
% ohpc_command (*\install*) ohpc-gnu14-mvapich2-parallel-libs
% ohpc_indent 0
% ohpc_command fi
% ohpc_command if [[ ${enable_opa} -eq 1 ]];then
% ohpc_indent 5
% ohpc_command (*\install*) ohpc-gnu14-mvapich2-parallel-libs
% ohpc_indent 0
% ohpc_command fi
% end_ohpc_run


\vspace*{.6cm}
\subsection{Optional Development Tool Builds} \label{sec:3rdparty_intel}
\input{common/oneapi_enabled_builds_slurm}

\section{Resource Manager Startup} \label{sec:rms_startup}
\input{common/slurm_startup_stateful}

\section{Run a Test Job} \label{sec:test_job}
\input{common/xcat_slurm_test_job}

\clearpage
\appendix
{\bf \LARGE \centerline{Appendices}} \vspace*{0.2cm}

\addcontentsline{toc}{section}{Appendices}
\renewcommand{\thesubsection}{\Alph{subsection}}

\input{common/automation_appendix}
\input{common/upgrade_stateful}
\input{common/test_suite}
\input{common/customization_appendix_centos}
../../warewulf/slurm/manifest.tex
\input{common/signature}


\end{document}

