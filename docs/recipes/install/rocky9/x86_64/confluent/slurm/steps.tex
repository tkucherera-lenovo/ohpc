\documentclass[letterpaper]{article}
\usepackage{common/ohpc-doc}
\setcounter{secnumdepth}{5}
\setcounter{tocdepth}{5}

% Include git variables
\input{vc.tex}

% Define Base OS and other local macros
\newcommand{\baseOS}{Rocky 9.4}
\newcommand{\OSRepo}{Rocky\_9.4}
\newcommand{\OSTree}{EL\_9}
\newcommand{\OSTag}{el9}
\newcommand{\baseos}{rocky9.4}
\newcommand{\baseosshort}{rocky9}
\newcommand{\provisioner}{confluent}
\newcommand{\provheader}{\provisioner{}}
\newcommand{\rms}{SLURM}
\newcommand{\rmsshort}{slurm}
\newcommand{\arch}{x86\_64}

% Define package manager commands
\newcommand{\pkgmgr}{dnf}
\newcommand{\addrepo}{wget -P /etc/yum.repos.d}
\newcommand{\chrootaddrepo}{wget -P \$CHROOT/etc/yum.repos.d}
\newcommand{\clean}{dnf clean expire-cache}
\newcommand{\chrootclean}{dnf --installroot=\$CHROOT clean expire-cache}
\newcommand{\install}{dnf -y install}
\newcommand{\chrootinstall}{dnf -y --installroot=\$CHROOT install}
\newcommand{\groupinstall}{dnf -y groupinstall}
\newcommand{\groupchrootinstall}{dnf -y --installroot=\$CHROOT groupinstall}
\newcommand{\remove}{dnf -y remove}
\newcommand{\upgrade}{dnf -y upgrade}
\newcommand{\chrootupgrade}{dnf -y --installroot=\$CHROOT upgrade}
\newcommand{\tftppkg}{syslinux-tftpboot}
\newcommand{\beegfsrepo}{https://www.beegfs.io/release/beegfs\_7.2.1/dists/beegfs-rhel8.repo}

% boolean for os-specific formatting
\toggletrue{isCentOS}
\toggletrue{isCentOS_ww_slurm_x86}
\toggletrue{isSLURM}
\toggletrue{isx86}
\toggletrue{isCentOS_x86}

\begin{document}
\graphicspath{{common/figures/}}
\thispagestyle{empty}

% Title Page
\input{common/title}
% Disclaimer
\input{common/legal}

\newpage
\tableofcontents
\newpage

% Introduction  --------------------------------------------------

\section{Introduction} \label{sec:introduction}
\input{common/install_header}
\input{common/intro} \\

\input{common/base_edition/edition}
\input{common/audience}
\input{common/requirements}
\input{common/inputs}



% Base Operating System --------------------------------------------

\section{Install Base Operating System (BOS)}
\input{common/bos}

%\clearpage
% begin_ohpc_run
% ohpc_validation_newline
% ohpc_validation_comment Disable firewall
\begin{lstlisting}[language=bash,keywords={}]
[sms](*\#*) systemctl disable firewalld
[sms](*\#*) systemctl stop firewalld
\end{lstlisting}
% end_ohpc_run

% ------------------------------------------------------------------

\section{Install \Confluent{} and Provision Nodes with BOS} \label{sec:provision_compute_bos}
Installation is accomplished in two steps: First,  a generic OS
image is installed on {\em compute} nodes and then, once the nodes are up
and running, \OHPC{} components are added to both the SMS and the nodes at the
same time.

\subsection{Enable \Confluent{} repository for local use} \label{sec:enable_confluent}
To begin, enable use of the public \Confluent{} repository by adding it to the local list
of available package repositories. This also requires network access from
your {\em master} server to the internet, or alternatively, that
the repository be mirrored locally. In this case, we use the network.

% begin_ohpc_run
% ohpc_validation_newline
% ohpc_comment_header Enable Confluent repositories \ref{sec:enable_confluent}
\begin{lstlisting}[language=bash,keywords={},basicstyle=\fontencoding{T1}\fontsize{8.0}{10}\ttfamily,
	literate={VER}{\OHPCVerTree{}}1 {OSREPO}{\OSTree{}}1 {TAG}{\OSTag{}}1 {ARCH}{\arch{}}1 {-}{-}1]
[sms](*\#*) (*\install*) dnf-plugins-core
[sms](*\#*) (*\addrepo*) https://hpc.lenovo.com/yum/latest/el9/x86_64/lenovo-hpc-yum-1-1.x86_64.rpm
\end{lstlisting}
% end_ohpc_run

\subsection{Add provisioning services on {\em master} node} \label{sec:add_provisioning}
% -*- mode: latex; fill-column: 120; -*-
With \Confluent{} repository enabled, issue the following install the provisioning
service on {\em master} node

% begin_ohpc_run
% ohpc_comment_header Add baseline OpenHPC and provisioning services \ref{sec:add_provisioning}
\begin{lstlisting}[language=bash,keywords={}]
[sms](*\#*) (*\install*) lenovo-confluent
[sms](*\#*) (*\install*) tftp-server

# enable Confluent and its tools for use in current shell
[sms](*\#*) systemctl enable confluent --now
[sms](*\#*) systemctl enable httpd --now
[sms](*\#*) systemctl enable tftp.socket --now
[sms](*\#*) source /etc/profile.d/confluent_env.sh
\end{lstlisting}
% ohpc_validation_newline
% end_ohpc_run

\vspace*{-0.15cm}
\subsection{Complete basic \Confluent{} setup for {\em master} node} \label{sec:setup_confluent}
At this point, all of the packages necessary to use \Confluent{} on the {\em master}
host should be installed. Next, we enable support for local provisioning using
a second private interface (refer to Figure~\ref{fig:physical_arch})

% begin_ohpc_run
% ohpc_comment_header Complete basic Confluent setup for master node \ref{sec:setup_confluent}
%\begin{verbatim}
\begin{lstlisting}[language=bash,literate={-}{-}1,keywords={},upquote=true,keepspaces]
# Enable internal interface for provisioning
[sms](*\#*) ip link set dev ${sms_eth_internal} up
[sms](*\#*) ip address add ${sms_ip}/${internal_netmask} broadcast + dev ${sms_eth_internal}

\end{lstlisting}
%\end{verbatim}
% end_ohpc_run


\noindent \Confluent{} requires a network domain name specification for system-wide name
resolution. This value can be set to match your local DNS schema or given a
unique identifier such as `local`. A default group called everything is 
automatically added to every node. It provides a method to indicate global settings.
Attributes may all be specified on the command line, and an example set could be:

% begin_ohpc_run
% ohpc_validation_newline
% ohpc_validation_comment Define local domainname, deployment protocol and dns 
\begin{lstlisting}[language=bash,keywords={},upquote=true,basicstyle=\footnotesize\ttfamily,literate={BOSVER}{\baseos{}}1]
[sms](*\#*) nodegroupattrib everything deployment.useinsecureprotocols=${deployment_protocols} dns.domain=${dns_domain} 
[sms](*\#*) nodegroupattrib everything dns.servers=${dns_servers} net.ipv4_gateway=${ipv4_gateway}
\end{lstlisting}

\noindent We will also define 

\subsection{Define {\em compute} image for provisioning}
% -*- mode: latex; fill-column: 120; -*-

With the provisioning services enabled, the next step is to define
a system image that can subsequently be
used to provision one or more {\em compute} nodes. The following subsections highlight this process.

\subsubsection{Build initial BOS image} \label{sec:assemble_bos}
The following steps illustrate the process to build a minimal, default image for use with \Confluent{}. To begin, you will
first need to have a local copy of the ISO image available for the underlying OS. In this recipe, the relevant ISO image
is \texttt{Rocky-9.4-x86\_64-dvd.iso} (available from the Rocky
\href{https://rockylinux.org/download/}{\color{blue}download} page).
We initialize the image
creation process using the \texttt{osdeploy} command assuming that the necessary ISO image is available locally in
\texttt{\$\{iso\_path\}} as follows:

The \texttt{osdeploy initialize} command is used to prepare a confluent server to deploy deploy operating systems.
For first time setup, run osdeploy initialize interactively to be walked through the various options using: 
\texttt{osdeploy initialize -i}

% begin_ohpc_run
% ohpc_comment_header Initialize OS images for use with Confluent \ref{sec:assemble_bos}
\begin{lstlisting}[language=bash,literate={-}{-}1,keywords={},upquote=true,keepspaces,literate={BOSVER}{\baseos{}}1]
[sms](*\#*) osdeploy initialize -${initialize_options}
[sms](*\#*) osdeploy import ${iso_path}

\end{lstlisting}
% end_ohpc_run

\noindent Once completed, OS image should be available for use within \Confluent{}. These can be queried via:

\begin{lstlisting}[language=bash,literate={-}{-}1,keywords={},upquote=true,keepspaces,literate={BOSVER}{\baseos{}}1]
# Query available images
[sms](*\#*) osdeploy list
Distributions:
  rocky-8.5-x86_64
  rocky-9.4-x86_64
Profiles:
  rhel-9.4-x86_64-default
  rocky-8.5-x86_64-default
\end{lstlisting}

If needing to copy files from the sms node to the compute nodes during deployment, this can be done by
modifying the syncfiles file that is created when \texttt{osdeploy import} command is run. For an environment
that has no DNS server and needs to have /etc/hosts file synced amongst all the nodes, the following command
should be run.   

% begin_ohpc_run
% ohpc_validation_newline
% ohpc_validation_comment Sync the hosts file in cluster
\begin{lstlisting}[language=bash,literate={-}{-}1,keywords={},upquote=true,keepspaces,literate={BOSVER}{\baseos{}}1]
  [sms](*\#*) echo "/etc/hosts -> /etc/hosts" >> /var/lib/confluent/public/os/rocky-9.4-x86_64-default/syncfiles

\end{lstlisting}
% end_ohpc_run
  

%The \texttt{CHROOT} environment variable highlights the path and is used by
%subsequent commands to augment the basic installation.



\vspace*{0.9cm}
\subsection{Add compute nodes into \Confluent{} database} \label{sec:confluent_add_nodes}
%\subsubsection{Register nodes for provisioning}

\noindent Next, we add {\em compute} nodes and define their properties as
attributes in \Confluent{} database.
These hosts are grouped logically into a group named {\em
compute} to facilitate group-level commands used later in the recipe. The compute
group has to be defined first before we can add any nodes to the group using the 
{\texttt nodegroup define} command. Note the
use of variable names for the desired compute hostnames, node IPs, MAC
addresses, and BMC login credentials, which should be modified to accommodate
local settings and hardware. To enable serial console access via  \Confluent{},
{\texttt console.method}
property is also defined. 

% begin_ohpc_run
% ohpc_validation_newline
% ohpc_validation_comment Add hosts to cluster \ref{sec:confluent_add_nodes}
\begin{lstlisting}[language=bash,keywords={},upquote=true,basicstyle=\footnotesize\ttfamily,]
#define the compute group
[sms](*\#*) nodegroupdefine compute

# Define nodes as objects in confluent database
[sms](*\#*) for ((i=0; i<$num_computes; i++)) ; do
nodedefine ${c_name[$i]} groups=everything,compute hardwaremanagement.manager=${c_bmc[$i]} \ 
secret.hardwaremanagementuser=${bmc_username} secret.hardwaremanagementpassword=${bmc_password} \ 
net.hwaddr=${c_mac[$i]}
        done
\end{lstlisting}
% end_ohpc_run

\begin{center}
  \begin{tcolorbox}[]
    \small
Defining nodes one-by-one, as done above, is only efficient
for a small number of nodes. For larger node counts,
\Confluent{} provides capabilities for automated detection and
configuration.
Consult the
\href{https://hpc.lenovo.com/users/documentation/confluentdisco.html}{\color{blue}\xCAT{}
Hardware Discovery \& Define Node Guide}.
\end{tcolorbox}
\end{center}

%\clearpage
If enabling {\em optional} IPoIB functionality (e.g. to support Lustre over \InfiniBand{}), additional
settings are required to define the IPoIB network with \Confluent{} and specify
desired IP settings for each compute. This can be accomplished as follows for
the {\em ib0} interface:

% begin_ohpc_run
% ohpc_validation_newline
% ohpc_validation_comment Setup IPoIB networking
% ohpc_command if [[ ${enable_ipoib} -eq 1 ]];then
% ohpc_indent 5
\begin{lstlisting}[language=bash,keywords={},upquote=true,basicstyle=\footnotesize\ttfamily]
# Register desired IPoIB IPs per compute
[sms](*\#*) for ((i=0; i<$num_computes; i++)) ; do
		nodeattrib ${c_name[i]} net.ib0.ipv4_address=${c_ipoib[i]}/${ipoib_netmask}
        done
\end{lstlisting}
% ohpc_indent 0
% ohpc_command fi
% end_ohpc_run

%\clearpage
With the desired compute nodes and domain identified, the remaining steps in the
provisioning configuration process are to define the provisioning mode and
image for the {\em compute} group and use \Confluent{} commands to complete
configuration for network services like DNS and DHCP. These tasks are
accomplished as follows:

%\clearpage
% begin_ohpc_run
% ohpc_validation_newline
% ohpc_validation_comment Complete networking setup, associate provisioning image
\begin{lstlisting}[language=bash,keywords={},upquote=true,basicstyle=\footnotesize\ttfamily,literate={BOSSHORT}{\baseosshort{}}1 {IMAGE}{\installimage{}}1]
# Associate desired provisioning image for computes
[sms](*\#*) nodedeploy -n compute -p rocky-9.4-x86_64-default
\end{lstlisting}

%%% If the Lustre client was enabled for computes in \S\ref{sec:lustre_client}, you
%%% should be able to mount the file system post-boot using the fstab entry
%%% (e.g. via ``\texttt{mount /mnt/lustre}''). Alternatively, if
%%% you prefer to have the file system mounted automatically at boot time, a simple
%%% postscript can be created and registered with \xCAT{} for this purpose as follows.
%%%
%%% % begin_ohpc_run
%%% % ohpc_validation_newline
%%% % ohpc_validation_comment Optionally create xCAT postscript to mount Lustre client
%%% % ohpc_command if [ ${enable_lustre_client} -eq 1 ];then
%%% % ohpc_indent 5
%%% \begin{lstlisting}[language=bash,keywords={},upquote=true,basicstyle=\footnotesize\ttfamily,literate={BOSVER}{\baseos{}}1]
%%% # Optionally create postscript to mount Lustre client at boot
%%% [sms](*\#*) echo '#!/bin/bash' > /install/postscripts/lustre-client
%%% [sms](*\#*) echo 'mount /mnt/lustre' >> /install/postscripts/lustre-client
%%% [sms](*\#*) chmod 755 /install/postscripts/lustre-client
%%% # Register script for computes
%%% [sms](*\#*) chdef compute -p postscripts=lustre-client
%%% \end{lstlisting}
%%% % ohpc_indent 0
%%% % ohpc_command fi
%%% % end_ohpc_run
%%%



%\vspace*{-0.25cm}
\subsection{Boot compute nodes} \label{sec:boot_computes}
Prior to booting the {\em compute} hosts, we configure them to use PXE as their
next boot mode. After the initial PXE, ensuing boots will return to using the default boot device
specified in the BIOS.

% begin_ohpc_run
% ohpc_comment_header Set nodes to netboot \ref{sec:boot_computes}
\begin{lstlisting}[language=bash,keywords={},upquote=true]
[sms](*\#*) nodesetboot compute network
\end{lstlisting}
% end_ohpc_run

At this point, the {\em master} server should be able to boot the newly defined
compute nodes. This is done by using the \texttt{nodepower} \Confluent{} command
leveraging IPMI protocol set up during the the {\em compute} node definition
in \S~\ref{sec:confluent_add_nodes}. The following power cycles each of the
desired hosts.


% begin_ohpc_run
% ohpc_comment_header Boot compute nodes
\begin{lstlisting}[language=bash,keywords={},upquote=true]
[sms](*\#*) nodepower compute boot
\end{lstlisting}
% end_ohpc_run

Once kicked off, the boot process should take about 5-10
minutes (depending on BIOS post times).  You can monitor the
provisioning by using the \texttt{nodeconsole} command, which displays serial console
for a selected node. Note that the escape sequence
is \texttt{CTRL-e c .} typed sequentially.

Successful provisioning can be verified by a parallel command on the compute
nodes. The \Confluent{}-provided
\texttt{nodeshell} command, which uses \Confluent{} node names and groups.  
For example, to run a command on
the newly imaged compute hosts using \texttt{nodeshell}, execute the following:

\begin{lstlisting}[language=bash]
[sms](*\#*) nodeshell compute uptime
c1:  12:56:50 up 14 min,  0 users,  load average: 0.00, 0.01, 0.04
c2:  12:56:50 up 13 min,  0 users,  load average: 0.00, 0.02, 0.05
c3:  12:56:50 up 14 min,  0 users,  load average: 0.00, 0.02, 0.05
c4:  12:56:50 up 14 min,  0 users,  load average: 0.00, 0.01, 0.04
\end{lstlisting}


\section{Install \OHPC{} Components} \label{sec:basic_install}
\input{common/install_ohpc_components_intro}

\subsection{Enable \OHPC{} repository for local use} \label{sec:enable_repo}
\OHPC{} provides an online package repository, but here we setup and use a local
mirror on the SMS host for \OHPC{} package installation. This is preferable for
a stateful install, because {\em compute} nodes can access it on the local
network.  


Assuming that the {\em master} has outside network access, the mirror is
created  by downloading a tarball of the repository, unpacking it and running a
provided script to configure the package manager.

% begin_ohpc_run
% ohpc_validation_newline
% ohpc_comment_header Download OHPC repo and create local mirror \ref{sec:enable_repo}
\begin{lstlisting}[language=bash,keywords={},basicstyle=\fontencoding{T1}\fontsize{7.6}{10}\ttfamily,
	literate={VER}{\OHPCVerTree{}}1 {OSTREE}{\OSTree{}}1 {TAG}{\OSTag{}}1 {ARCH}{\arch{}}1 {-}{-}1 
        {VERLONG}{\OHPCVersion{}}1]
# Download OHPC tarball for local installation
[sms](*\#*) wget http://repos.openhpc.community/dist/VERLONG/OpenHPC-VERLONG.OSTREE.ARCH.tar
# Create directory if necessary
[sms](*\#*) mkdir -p $ohpc_repo_dir
#Unpack
[sms](*\#*) tar xvf OpenHPC-VERLONG.OSTREE.ARCH.tar -C $ohpc_repo_dir
#Create local mirror repository
[sms](*\#*) $ohpc_repo_dir/make_repo.sh
\end{lstlisting}
% end_ohpc_run


% begin_ohpc_run
% ohpc_validation_newline
% ohpc_validation_comment Verify OpenHPC repository has been enabled before proceeding
% ohpc_validation_newline
% ohpc_command dnf repolist | grep -q OpenHPC
% ohpc_command if [ $? -ne 0 ];then
% ohpc_command    echo "Error: OpenHPC repository must be enabled locally"
% ohpc_command    exit 1
% ohpc_command fi
% end_ohpc_run

\input{common/rocky_repos}


% begin_ohpc_run
\begin{lstlisting}[language=bash,keywords={},basicstyle=\fontencoding{T1}\fontsize{8.0}{10}\ttfamily,literate={ARCH}{\arch{}}1 {-}{-}1]
[sms](*\#*)  (*\install*) epel-release
\end{lstlisting}
% end_ohpc_run

Now \OHPC{} packages can be installed. To add the base package on the SMS
issue the following
% begin_ohpc_run
\begin{lstlisting}[language=bash,keywords={},basicstyle=\fontencoding{T1}\fontsize{8.0}{10}\ttfamily,literate={ARCH}{\arch{}}1 {-}{-}1]
[sms](*\#*)  (*\install*) ohpc-base
\end{lstlisting}
% end_ohpc_run

\input{common/automation}

\subsection{Setup time synchronization service on {\em master} node} \label{sec:add_ntp}
\input{common/time}

%\input{common/enable_pxe}


\subsection{Add resource management services on {\em master} node} \label{sec:add_rm}
\input{common/install_slurm}

\subsection{Optionally add \InfiniBand{} support services on {\em master} node} \label{sec:add_ofed}
\input{common/ibsupport_sms_centos}

\subsection{Optionally add \OmniPath{} support services on {\em master} node} \label{sec:add_opa}
\input{common/opasupport_sms_centos}

\vspace*{0.2cm}
\subsubsection{Add \OHPC{} components} \label{sec:add_components}
% -*- mode: latex; fill-column: 120; -*-

The next step is adding \OHPC{} components to the {\em compute} nodes that at this
point are running basic OSes.  This process will leverage two \Confluent{}-provided
commands: \texttt{nodeshell} to run \texttt{\pkgmgr{}} installer on all the
nodes in parallel  and \texttt{nodersync} to distribute configuration files from the
SMS to the {\em compute} nodes.

\noindent To do this, repositories on the {\em compute} nodes need to be configured
properly.

\Confluent{} has automatically setup an  OS repository on the SMS and configured the
nodes to use it, but it has  also enabled online OS repositories.


\noindent Next, we alse add the OHPC repo to the compute nodes \S\ref{sec:enable_repo}

% begin_ohpc_run
% ohpc_comment_header Setup nodes repositories and Install OHPC components \ref{sec:add_components}
\begin{lstlisting}[language=bash,literate={-}{-}1,keywords={},upquote=true]
# Add OpenHPC repo 
[sms](*\#*) (*\chrootinstall*) http://repos.openhpc.community/OpenHPC/3/EL_9/x86_64/ohpc-release-3-1.el9.x86_64.rpm"
\end{lstlisting}
% end_ohpc_run

The {\em compute} nodes also need access to the EPEL repository, a required
dependency for \OHPC{} packages. 

% begin_ohpc_run
% ohpc_comment_header Configure access to EPEL repo
\begin{lstlisting}[language=bash,literate={-}{-}1,keywords={},upquote=true]
# Add epel repo
[sms](*\#*) nodeshell compute "dnf -y  install epel-release"

\end{lstlisting}
% end_ohpc_run


\noindent Additionally, a workaround is needed for \OHPC{} documentation files,
which are installed into a read-only NFS share /opt/ohpc/pub. Any package
attempting to write to that directory will fail to install. The following
prevents that by directing \texttt{rpm} not to install documentation files on
the {\em compute} nodes:

% begin_ohpc_run
\begin{lstlisting}[language=bash,literate={-}{-}1,keywords={},upquote=true]
[sms](*\#*) nodeshell compute echo -e %_excludedocs 1 \>\> ~/.rpmmacros
\end{lstlisting}
% end_ohpc_run

\noindent Now \OHPC{} and other cluster-related software components can be
installed on the nodes. The first step is to install a base compute package:
% begin_ohpc_run
% ohpc_comment_header Add OpenHPC base components to compute image
\begin{lstlisting}[language=bash,literate={-}{-}1,keywords={},upquote=true]
# Install compute node base meta-package
[sms](*\#*) (*\chrootinstall*) ohpc-base-compute
\end{lstlisting}
% end_ohpc_run

\noindent Next, we can include additional components:


%\newpage
% begin_ohpc_run
% ohpc_validation_comment Add OpenHPC components to compute instance
\begin{lstlisting}[language=bash,literate={-}{-}1,keywords={},upquote=true]
# Add Slurm client support meta-package
[sms](*\#*) (*\chrootinstall*) ohpc-slurm-client

# Add Network Time Protocol (NTP) support
[sms](*\#*) (*\chrootinstall*) ntp

# Add kernel drivers
[sms](*\#*) (*\chrootinstall*) kernel

# Include modules user environment
[sms](*\#*) (*\chrootinstall*)  --enablerepo=powertools lmod-ohpc
\end{lstlisting}
% end_ohpc_run

% ohpc_comment_header Optionally add InfiniBand support services in compute node image \ref{sec:add_components}
% ohpc_command if [[ ${enable_ib} -eq 1 ]];then
% ohpc_indent 5
\begin{lstlisting}[language=bash,literate={-}{-}1,keywords={},upquote=true]
# Optionally add IB support and enable
[sms](*\#*) (*\groupchrootinstall*) "InfiniBand Support"
\end{lstlisting}
% ohpc_indent 0
% ohpc_command fi
% end_ohpc_run

\vspace*{-0.25cm}
\subsubsection{Customize system configuration} \label{sec:master_customization}
Here we set up \NFS{}  mounting of a
\$HOME file system and the public \OHPC{} install path (\texttt{/opt/ohpc/pub})
that will be hosted by the {\em master} host in this  example configuration.

\vspace*{0.15cm}
% begin_ohpc_run
% ohpc_comment_header Customize system configuration \ref{sec:master_customization}
\begin{lstlisting}[language=bash,literate={-}{-}1,keywords={},upquote=true]
# Disable /tftpboot and /install export entries
[sms](*\#*) perl -pi -e "s|/tftpboot|#/tftpboot|" /etc/exports
[sms](*\#*) perl -pi -e "s|/install|#/install|" /etc/exports

# Export /home and OpenHPC public packages from master server
[sms](*\#*) echo "/home *(rw,no_subtree_check,fsid=10,no_root_squash)" >> /etc/exports
[sms](*\#*) echo "/opt/ohpc/pub *(ro,no_subtree_check,fsid=11)" >> /etc/exports
[sms](*\#*) exportfs -a
[sms](*\#*) systemctl restart nfs-server
[sms](*\#*) systemctl enable nfs-server

# Create NFS client mounts of /home and /opt/ohpc/pub on compute hosts
[sms](*\#*) nodeshell compute echo \
        "\""${sms_ip}:/home /home nfs nfsvers=3,nodev,nosuid 0 0"\"" \>\> /etc/fstab
[sms](*\#*) nodeshell compute echo \
        "\""${sms_ip}:/opt/ohpc/pub /opt/ohpc/pub nfs nfsvers=3,nodev 0 0"\"" \>\> /etc/fstab
[sms](*\#*) nodeshell compute systemctl restart nfs

# Mount NFS shares
[sms](*\#*) nodeshell compute mount /home
[sms](*\#*) nodeshell compute mkdir -p /opt/ohpc/pub
[sms](*\#*) nodeshell compute mount /opt/ohpc/pub

\end{lstlisting}
% end_ohpc_run



% Additional commands when additional computes are requested

% begin_ohpc_run
% ohpc_validation_newline
% ohpc_validation_comment Update basic slurm configuration if additional computes defined
% ohpc_validation_comment This is performed on the SMS, nodes will pick it up config file is copied there later
% ohpc_command if [ ${num_computes} -gt 4 ];then
% ohpc_command    perl -pi -e "s/^NodeName=(\S+)/NodeName=${compute_prefix}[1-${num_computes}]/" /etc/slurm/slurm.conf
% ohpc_command    perl -pi -e "s/^PartitionName=normal Nodes=(\S+)/PartitionName=normal Nodes=${compute_prefix}[1-${num_computes}]/" /etc/slurm/slurm.conf
% ohpc_command fi
% end_ohpc_run

%\clearpage
\subsubsection{Additional Customization ({\em optional})} \label{sec:addl_customizations}
\input{common/compute_customizations_intro}

\paragraph{Increase locked memory limits}
\input{common/memlimits_stateful}

\paragraph{Enable ssh control via resource manager}
\input{common/slurm_pam_stateful}

\paragraph{Add \Lustre{} client} \label{sec:lustre_client}
\input{common/lustre-client}
\input{common/lustre-client-centos-stateful}
\input{common/lustre-client-post-stateful}

\vspace*{0.4cm}

\paragraph{Add \clustershell{}}
\input{common/clustershell}

\paragraph{Add \genders{}}
\input{common/genders}

\paragraph{Add Magpie}
\input{common/magpie}

\paragraph{Add \conman{}} \label{sec:add_conman}
\input{common/conman}

\paragraph{Add \nhc{}} \label{sec:add_nhc}
\input{common/nhc}
\input{common/nhc_slurm}

%\subsubsection{Identify files for synchronization} \label{sec:file_import}
%The \Confluent{} system includes functionality to synchronize files located on the
SMS server for distribution to managed hosts. This is one way to
distribute user credentials to {\em compute} nodes (alternatively, you may
prefer to use a central authentication service like LDAP). To import local file-based
credentials, issue the following to enable the {\em synclist} feature and
register user credential files:

% begin_ohpc_run
% ohpc_comment_header Import files \ref{sec:file_import}
\begin{lstlisting}[language=bash,literate={-}{-}1,keywords={},upquote=true,literate={BOSSHORT}{\baseosshort{}}1]
# Add desired credential files to synclist
[sms](*\#*) echo "/etc/passwd -> /etc/passwd" > /var/lib/confluent/public/os/rocky-9.4-x86_64-default
[sms](*\#*) echo "/etc/group -> /etc/group" >> /var/lib/confluent/public/os/rocky-9.4-x86_64-default
[sms](*\#*) echo "/etc/shadow -> /etc/shadow" >> /var/lib/confluent/public/os/rocky-9.4-x86_64-default
\end{lstlisting}
% \end_ohpc_run
%\noindent Similarly, to import the
%global Slurm configuration file and the
cryptographic
key
%and associated file permissions
that is required by the {\em munge}
authentication library to be available on every host in the resource management
pool, issue the following:

% begin_ohpc_run
\begin{lstlisting}[language=bash,literate={-}{-}1,keywords={},upquote=true]
[sms](*\#*) echo "/etc/munge/munge.key -> /etc/munge/munge.key" >> /var/lib/confluent/public/os/rocky-9.4-x86_64-default
\end{lstlisting}
% \end_ohpc_run

\begin{center}
\begin{tcolorbox}[]
\small
The ``\texttt{nodeapply compute -F}'' command can be used to distribute changes made to any
defined synchronization files on the SMS host. Users wishing to automate this process may
want to consider adding a crontab entry to perform this action at defined intervals.
\end{tcolorbox}
\end{center}

%%%\subsubsection{Optional kernel arguments} \label{sec:optional_kargs}
%%%\input{common/conman_post}

\section{Install \OHPC{} Development Components}
\input{common/dev_intro}

%\vspace*{-0.15cm}
%\clearpage
\subsection{Development Tools} \label{sec:install_dev_tools}
\input{common/dev_tools}

\vspace*{-0.15cm}
\subsection{Compilers} \label{sec:install_compilers}
\OHPC{} presently packages the \GNU{} compiler toolchain integrated with the
underlying Lmod modules system in a hierarchical fashion. The modules
system will conditionally present compiler-dependent software based on the
toolchain currently loaded.

% begin_ohpc_run
% ohpc_comment_header Install Compilers \ref{sec:install_compilers}
\begin{lstlisting}[language=bash]
[sms](*\#*) (*\install*) gnu14-compilers-ohpc
\end{lstlisting}
% end_ohpc_run

%%% The llvm compiler toolchains are also provided as a standalone additional
%%% compiler family (ie. users can easily switch between gcc/clang environments),
%%% but we do not provide the full complement of downstream library builds.
%%%
%%% % begin_ohpc_run
%%% % ohpc_comment_header Install llvm Compilers
%%% \begin{lstlisting}[language=bash]
%%% [sms](*\#*) (*\install*) llvm5-compilers-ohpc
%%% \end{lstlisting}
%%% % end_ohpc_run


%\clearpage
\subsection{MPI Stacks} \label{sec:mpi}
For MPI development and runtime support, \OHPC{} provides pre-packaged builds
for a variety of MPI families and transport layers. Currently available options
and their applicability to various network transports are summarized in
Table~\ref{table:mpi}.  The command that follows installs a starting set of MPI
families that are compatible with both ethernet and high-speed fabrics.

\iftoggleverb{isx86}
% x86_64

\begin{table}[h]
\caption{Available MPI variants} \label{table:mpi}
\centering
\begin{tabular}{@{\hspace*{0.2cm}} *5l @{}}    \toprule
                                  & Ethernet (TCP)                 & \InfiniBand{}                  & \IntelR{} Omni-Path            \\ \midrule
\rowcolor{black!10} MPICH (ofi) & \multicolumn{1}{c}{\checkmark} & \multicolumn{1}{c}{\checkmark} & \multicolumn{1}{c}{\checkmark} \\
 MPICH (ucx)       & \multicolumn{1}{c}{\checkmark} & \multicolumn{1}{c}{\checkmark} & \multicolumn{1}{c}{\checkmark} \\
\rowcolor{black!10} MVAPICH2                          &                                & \multicolumn{1}{c}{\checkmark} &                                \\
MVAPICH2 (psm2) &                              &                                & \multicolumn{1}{c}{\checkmark} \\
\rowcolor{black!10} OpenMPI (ofi/ucx)            & \multicolumn{1}{c}{\checkmark} & \multicolumn{1}{c}{\checkmark} & \multicolumn{1}{c}{\checkmark} \\
%\rowcolor{black!10} OpenMPI (PMIx) & \multicolumn{1}{c}{\checkmark} & \multicolumn{1}{c}{\checkmark} & \multicolumn{1}{c}{\checkmark} \\ \bottomrule
\end{tabular}
\end{table}

\else
% aarch64

\begin{table}[h]
\caption{Available MPI builds} \label{table:mpi}
\centering
\begin{tabular}{@{\hspace*{0.2cm}} *5l @{}}    \toprule
                                  & Ethernet (TCP)                 & \InfiniBand{}                              \\ \midrule
\rowcolor{black!10} MPICH         & \multicolumn{1}{c}{\checkmark} &                                            \\
\rowcolor{black!10} OpenMPI                           & \multicolumn{1}{c}{\checkmark} & \multicolumn{1}{c}{\checkmark} \\
\end{tabular}
\end{table}

\fi

% begin_ohpc_run
% ohpc_comment_header Install MPI Stacks \ref{sec:mpi}
% ohpc_command if [[ ${enable_mpi_defaults} -eq 1 ]];then
% ohpc_indent 5
\begin{lstlisting}[language=bash]
[sms](*\#*) (*\install*) openmpi5-pmix-gnu14-ohpc mpich-ofi-gnu14-ohpc
\end{lstlisting}
% ohpc_indent 0
% ohpc_command fi
% end_ohpc_run

Note that OpenHPC 2.x introduces the use of two related transport layers for
the MPICH and OpenMPI builds that support a variety of underlying
fabrics: \href{https://www.openucx.org}{UCX} (Unified Communication X)
and \href{https://ofiwg.github.io/libfabric/}{OFI} (OpenFabrics interfaces).
In the case of OpenMPI, a monolithic build is provided which supports both
transports and end-users can customize their runtime preferences with
environment variables. For MPICH, two separate builds are provided and the
example above highlighted installing the {\texttt ofi} variant.  However, the
packaging is designed such that both versions can be installed simultaneously
and users can switch between the two via normal module command
semantics. Alternatively, a site can choose to install the {\texttt ucx} variant
instead as a drop-in MPICH replacement:

\begin{lstlisting}[language=bash]
[sms](*\#*) (*\install*) mpich-ucx-gnu14-ohpc
\end{lstlisting}

In the case where both MPICH variants are installed, two modules will be
visible in the end-user environment and an example of this configuration is
highlighted is below.

\begin{lstlisting}[language=bash]
[sms](*\#*) module avail mpich

-------------------- /opt/ohpc/pub/moduledeps/gnu14---------------------
   mpich/3.4.3-ofi    mpich/3.4.3-ucx (D)
\end{lstlisting}

If your system includes \InfiniBand{} and you enabled underlying support in
\S\ref{sec:add_ofed} and \S\ref{sec:addl_customizations}, an additional
MVAPICH2 family is available for use:

% begin_ohpc_run
% ohpc_validation_newline
% ohpc_command if [[ ${enable_ib} -eq 1 ]];then
% ohpc_indent 5
\begin{lstlisting}[language=bash]
[sms](*\#*) (*\install*) mvapich2-gnu14-ohpc
\end{lstlisting}
% ohpc_indent 0
% ohpc_command fi
% end_ohpc_run

Alternatively, if your system includes \IntelR{} \OmniPath{}, use the (\texttt{psm2})
variant of MVAPICH2 instead:

% begin_ohpc_run
% ohpc_command if [[ ${enable_opa} -eq 1 ]];then
% ohpc_indent 5
\begin{lstlisting}[language=bash]
[sms](*\#*) (*\install*) mvapich2-psm2-gnu14-ohpc
\end{lstlisting}
% ohpc_indent 0
% ohpc_command fi
% end_ohpc_run

%%--
%% https://github.com/openhpc/ohpc/issues/1273
%% disabling until we can get pmix/openmpi/slurm to play nicely
%%--
%% An additional OpenMPI build variant is listed in Table~\ref{table:mpi} which
%% enables \href{https://pmix.github.io/pmix/}{\color{blue}{PMIx}} job launch
%% support for use with \SLURM{}. This optional variant is
%% available as \texttt{openmpi5-pmix-slurm-gnu9-ohpc}.


\subsection{Performance Tools} \label{sec:install_perf_tools}
\OHPC{} provides a variety of open-source tools to aid in application
performance analysis (refer to Appendix~\ref{appendix:manifest} for a listing
of available packages). This group of tools can be installed as follows:

% begin_ohpc_run
% ohpc_comment_header Install Performance Tools \ref{sec:install_perf_tools}
\begin{lstlisting}[language=bash,keywords={},literate={-}{-}1]
# Install perf-tools meta-package
[sms](*\#*) (*\install*) ohpc-gnu14-perf-tools
\end{lstlisting}
% end_ohpc_run


\subsection{Setup default development environment}
System users often find it convenient to have a default development environment
in place so that compilation can be performed directly for parallel programs
requiring MPI. This setup can be conveniently enabled via modules and the \OHPC{}
modules environment is pre-configured to load an \texttt{ohpc} module on login
(if present). The following package install provides a default
environment that enables autotools, the \GNU{} compiler toolchain, and the
OpenMPI stack.

% begin_ohpc_run
\begin{lstlisting}[language=bash]
[sms](*\#*) (*\install*) lmod-defaults-gnu14-openmpi5-ohpc
\end{lstlisting}
% end_ohpc_run

\begin{center}
\begin{tcolorbox}[]
\small
\iftoggleverb{isx86}
If you want to change the default environment from the suggestion above, \OHPC{}
also provides the \GNU{} compiler toolchain with the MPICH and MVAPICH2 stacks:
\fi

\iftoggleverb{isaarch}
If you want to change the default environment from the suggestion above, \OHPC{}
also provides additional default options using the \GNU{} compiler toolchain
with multiple MPICH variants or MVAPICH2. Relevant lmod-default packages names
are as follows:
\fi

\begin{itemize*}
\item lmod-defaults-gnu14-mpich-ofi-ohpc
\item lmod-defaults-gnu14-mpich-ucx-ohpc
\iftoggleverb{isx86}
\item lmod-defaults-gnu14-mvapich2-ohpc
\fi
\end{itemize*}
\end{tcolorbox}
\end{center}


%\vspace*{0.2cm}
\subsection{3rd Party Libraries and Tools} \label{sec:3rdparty}
\input{common/third_party_libs_intro}

\begin{center}
\begin{tcolorbox}[]
\small
\OHPC{}-provided 3rd party builds are configured to be installed
into a common top-level repository so that they can be easily exported to
desired hosts within the cluster. This common top-level path
(\path{/opt/ohpc/pub}) was previously configured to be mounted on {\em
 compute} nodes in \S\ref{sec:master_customization}, so the packages will be
immediately available for use on the cluster after installation on the {\em
 master} host.
\end{tcolorbox}
\end{center}

%\iftoggle{isCentOS}{\clearpage}
%\nottoggle{isCentOS}{\clearpage}

For convenience, \OHPC{} provides package aliases for these 3rd party libraries
and utilities that can be used to install available libraries for use with the
GNU compiler family toolchain. For parallel libraries, aliases are grouped by
MPI family toolchain so that administrators can choose a subset should they
favor a particular MPI stack.  Please refer to Appendix~\ref{appendix:manifest}
for a more detailed listing of all available packages in each of these functional
areas. To install all available package offerings within \OHPC{}, issue the
following:

% begin_ohpc_run
% ohpc_comment_header Install 3rd Party Libraries and Tools \ref{sec:3rdparty}
\begin{lstlisting}[language=bash,keywords={},upquote=true,keepspaces]
# Install 3rd party libraries/tools meta-packages built with GNU toolchain
[sms](*\#*) (*\install*) ohpc-gnu14-serial-libs
[sms](*\#*) (*\install*) ohpc-gnu14-io-libs
[sms](*\#*) (*\install*) ohpc-gnu14-python-libs
[sms](*\#*) (*\install*) ohpc-gnu14-runtimes
\end{lstlisting}
% end_ohpc_run





% begin_ohpc_run
% ohpc_command if [[ ${enable_mpi_defaults} -eq 1 ]];then
% ohpc_indent 5
\begin{lstlisting}[language=bash,keywords={},upquote=true,keepspaces]
# Install parallel lib meta-packages for all available MPI toolchains
[sms](*\#*) (*\install*) ohpc-gnu14-mpich-parallel-libs
[sms](*\#*) (*\install*) ohpc-gnu14-openmpi5-parallel-libs
\end{lstlisting}
% ohpc_indent 0
% ohpc_command fi
% ohpc_command if [[ ${enable_ib} -eq 1 ]];then
% ohpc_indent 5
% ohpc_command (*\install*) ohpc-gnu14-mvapich2-parallel-libs
% ohpc_indent 0
% ohpc_command fi
% ohpc_command if [[ ${enable_opa} -eq 1 ]];then
% ohpc_indent 5
% ohpc_command (*\install*) ohpc-gnu14-mvapich2-parallel-libs
% ohpc_indent 0
% ohpc_command fi
% end_ohpc_run


\vspace*{.6cm}
\subsection{Optional Development Tool Builds} \label{sec:3rdparty_intel}
\input{common/oneapi_enabled_builds_slurm}

\section{Resource Manager Startup} \label{sec:rms_startup}
\input{common/slurm_startup_stateful}

\section{Run a Test Job} \label{sec:test_job}
With the resource manager enabled for production usage, users should now be
able to run jobs. To demonstrate this, we will add a ``test'' user on the {\em master}
host that can be used to run an example job.

% begin_ohpc_run
\begin{lstlisting}[language=bash,keywords={}]
[sms](*\#*) useradd -m test
\end{lstlisting}
% end_ohpc_run

Next, the user's credentials need to be distributed across the cluster.
\Confluent{}'s \texttt{nodeappy} has a merge functionality that adds new entries into
credential files on {\em compute} nodes:

% begin_ohpc_run
\begin{lstlisting}[language=bash,keywords={}]
# Create a sync file for pushing user credentials to the nodes
[sms](*\#*) echo "/etc/passwd -> /etc/passwd" >> /var/lib/confluent/public/os/rocky-9.4-x86_64-default/syncfiles
[sms](*\#*) echo "/etc/group -> /etc/group"   >> /var/lib/confluent/public/os/rocky-9.4-x86_64-default/syncfiles
[sms](*\#*) echo "/etc/shadow -> /etc/shadow" >> /var/lib/confluent/public/os/rocky-9.4-x86_64-default/syncfiles
# Use Confluent to distribute credentials to nodes
[sms](*\#*) nodeapply -F compute 
\end{lstlisting}
% end_ohpc_run


~\\
\input{common/prun}

\iftoggle{isSLES_ww_slurm_x86}{\clearpage}
%\iftoggle{isxCAT}{\clearpage}

\subsection{Interactive execution}
To use the newly created ``test'' account to compile and execute the
application {\em interactively} through the resource manager, execute the
following (note the use of \texttt{prun} for parallel job launch which summarizes
the underlying native job launch mechanism being used):

\begin{lstlisting}[language=bash,keywords={}]
# Switch to "test" user
[sms](*\#*) su - test

# Compile MPI "hello world" example
[test@sms ~]$ mpicc -O3 /opt/ohpc/pub/examples/mpi/hello.c

# Submit interactive job request and use prun to launch executable
[test@sms ~]$ salloc -n 8 -N 2

[test@c1 ~]$ prun ./a.out

[prun] Master compute host = c1
[prun] Resource manager = slurm
[prun] Launch cmd = mpiexec.hydra -bootstrap slurm ./a.out

 Hello, world (8 procs total)
    --> Process #   0 of   8 is alive. -> c1
    --> Process #   4 of   8 is alive. -> c2
    --> Process #   1 of   8 is alive. -> c1
    --> Process #   5 of   8 is alive. -> c2
    --> Process #   2 of   8 is alive. -> c1
    --> Process #   6 of   8 is alive. -> c2
    --> Process #   3 of   8 is alive. -> c1
    --> Process #   7 of   8 is alive. -> c2
\end{lstlisting}

\begin{center}
\begin{tcolorbox}[]
The following table provides approximate command equivalences between SLURM and
OpenPBS:

\vspace*{0.15cm}
\input common/rms_equivalence_table
\end{tcolorbox}
\end{center}
\nottoggle{isCentOS}{\clearpage}

\iftoggle{isCentOS}{\clearpage}

\subsection{Batch execution}

For batch execution, \OHPC{} provides a simple job script for reference (also
housed in the \path{/opt/ohpc/pub/examples} directory. This example script can
be used as a starting point for submitting batch jobs to the resource manager
and the example below illustrates use of the script to submit a batch job for
execution using the same executable referenced in the previous interactive example.

\begin{lstlisting}[language=bash,keywords={}]
# Copy example job script
[test@sms ~]$ cp /opt/ohpc/pub/examples/slurm/job.mpi .

# Examine contents (and edit to set desired job sizing characteristics)
[test@sms ~]$ cat job.mpi
#!/bin/bash

#SBATCH -J test               # Job name
#SBATCH -o job.%j.out         # Name of stdout output file (%j expands to %jobId)
#SBATCH -N 2                  # Total number of nodes requested
#SBATCH -n 16                 # Total number of mpi tasks #requested
#SBATCH -t 01:30:00           # Run time (hh:mm:ss) - 1.5 hours

# Launch MPI-based executable

prun ./a.out

# Submit job for batch execution
[test@sms ~]$ sbatch job.mpi
Submitted batch job 339
\end{lstlisting}

\begin{center}
\begin{tcolorbox}[]
\small
The use of the \texttt{\%j} option in the example batch job script shown is a convenient
way to track application output on an individual job basis. The \texttt{\%j} token
is replaced with the \SLURM{} job allocation number once assigned (job~\#339 in
this example).
\end{tcolorbox}
\end{center}

\clearpage
\appendix
{\bf \LARGE \centerline{Appendices}} \vspace*{0.2cm}

\addcontentsline{toc}{section}{Appendices}
\renewcommand{\thesubsection}{\Alph{subsection}}

\input{common/automation_appendix}
\input{common/upgrade_stateful}
\input{common/test_suite}
\input{common/customization_appendix_centos}
../../warewulf/slurm/manifest.tex
\input{common/signature}


\end{document}

    

