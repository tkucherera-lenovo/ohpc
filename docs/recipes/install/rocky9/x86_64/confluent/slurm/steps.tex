\documentclass[letterpaper]{article}
\usepackage{common/ohpc-doc}
\setcounter{secnumdepth}{5}
\setcounter{tocdepth}{5}

% Include git variables
\input{vc.tex}

% Define Base OS and other local macros
\newcommand{\baseOS}{Rocky 9.4}
\newcommand{\OSRepo}{Rocky\_9.4}
\newcommand{\OSTree}{EL\_9}
\newcommand{\OSTag}{el9}
\newcommand{\baseos}{rocky9.4}
\newcommand{\baseosshort}{rocky9}
\newcommand{\provisioner}{confluent}
\newcommand{\provheader}{\provisioner{}}
\newcommand{\rms}{SLURM}
\newcommand{\rmsshort}{slurm}
\newcommand{\arch}{x86\_64}

% Define package manager commands
\newcommand{\pkgmgr}{dnf}
\newcommand{\addrepo}{wget -P /etc/yum.repos.d}
\newcommand{\chrootaddrepo}{wget -P \$CHROOT/etc/yum.repos.d}
\newcommand{\clean}{dnf clean expire-cache}
\newcommand{\chrootclean}{dnf --installroot=\$CHROOT clean expire-cache}
\newcommand{\install}{dnf -y install}
\newcommand{\chrootinstall}{dnf -y --installroot=\$CHROOT install}
\newcommand{\groupinstall}{dnf -y groupinstall}
\newcommand{\groupchrootinstall}{dnf -y --installroot=\$CHROOT groupinstall}
\newcommand{\remove}{dnf -y remove}
\newcommand{\upgrade}{dnf -y upgrade}
\newcommand{\chrootupgrade}{dnf -y --installroot=\$CHROOT upgrade}
\newcommand{\tftppkg}{syslinux-tftpboot}
\newcommand{\beegfsrepo}{https://www.beegfs.io/release/beegfs\_7.2.1/dists/beegfs-rhel8.repo}

% boolean for os-specific formatting
\toggletrue{isCentOS}
\toggletrue{isCentOS_ww_slurm_x86}
\toggletrue{isSLURM}
\toggletrue{isx86}
\toggletrue{isCentOS_x86}

\begin{document}
\graphicspath{{common/figures/}}
\thispagestyle{empty}

% Title Page
\input{common/title}
% Disclaimer
\input{common/legal}

\newpage
\tableofcontents
\newpage

% Introduction  --------------------------------------------------

\section{Introduction} \label{sec:introduction}
\input{common/install_header}
\input{common/intro} \\

\input{common/base_edition/edition}
\input{common/audience}
\input{common/requirements}
\input{common/inputs}



% Base Operating System --------------------------------------------

\section{Install Base Operating System (BOS)}
\input{common/bos}

%\clearpage
% begin_ohpc_run
% ohpc_validation_newline
% ohpc_validation_comment Disable firewall
\begin{lstlisting}[language=bash,keywords={}]
[sms](*\#*) systemctl disable firewalld
[sms](*\#*) systemctl stop firewalld
\end{lstlisting}
% end_ohpc_run

% ------------------------------------------------------------------

\section{Install \Confluent{} and Provision Nodes with BOS} \label{sec:provision_compute_bos}
Installation is accomplished in two steps: First,  a generic OS
image is installed on {\em compute} nodes and then, once the nodes are up
and running, \OHPC{} components are added to both the SMS and the nodes at the
same time.


% begin_ohpc_run
% ohpc_validation_newline
% ohpc_validation_comment Verify OpenHPC repository has been enabled before proceeding
% ohpc_validation_newline
% ohpc_command dnf repolist | grep -q OpenHPC
% ohpc_command if [ $? -ne 0 ];then
% ohpc_command    echo "Error: OpenHPC repository must be enabled locally"
% ohpc_command    exit 1
% ohpc_command fi
% end_ohpc_run

\input{common/rocky_repos}


% begin_ohpc_run
\begin{lstlisting}[language=bash,keywords={},basicstyle=\fontencoding{T1}\fontsize{8.0}{10}\ttfamily,literate={ARCH}{\arch{}}1 {-}{-}1]
[sms](*\#*)  (*\install*) epel-release
\end{lstlisting}
% end_ohpc_run

Now \OHPC{} packages can be installed. To add the base package on the SMS
issue the following
% begin_ohpc_run
\begin{lstlisting}[language=bash,keywords={},basicstyle=\fontencoding{T1}\fontsize{8.0}{10}\ttfamily,literate={ARCH}{\arch{}}1 {-}{-}1]
[sms](*\#*)  (*\install*) ohpc-base
\end{lstlisting}
% end_ohpc_run

\input{common/automation}

\subsection{Setup time synchronization service on {\em master} node} \label{sec:add_ntp}
\input{common/time}

%\input{common/enable_pxe}


\subsection{Add resource management services on {\em master} node} \label{sec:add_rm}
\input{common/install_slurm}

\subsection{Optionally add \InfiniBand{} support services on {\em master} node} \label{sec:add_ofed}
\input{common/ibsupport_sms_centos}

\subsection{Optionally add \OmniPath{} support services on {\em master} node} \label{sec:add_opa}
\input{common/opasupport_sms_centos}

\vspace*{0.2cm}
\subsubsection{Add \OHPC{} components} \label{sec:add_components}
% -*- mode: latex; fill-column: 120; -*-

The next step is adding \OHPC{} components to the {\em compute} nodes that at this
point are running basic OSes.  This process will leverage two \Confluent{}-provided
commands: \texttt{nodeshell} to run \texttt{\pkgmgr{}} installer on all the
nodes in parallel  and \texttt{nodersync} to distribute configuration files from the
SMS to the {\em compute} nodes.

\noindent To do this, repositories on the {\em compute} nodes need to be configured
properly.

\Confluent{} has automatically setup an  OS repository on the SMS and configured the
nodes to use it, but it has  also enabled online OS repositories.


\noindent Next, we alse add the OHPC repo to the compute nodes \S\ref{sec:enable_repo}

% begin_ohpc_run
% ohpc_comment_header Setup nodes repositories and Install OHPC components \ref{sec:add_components}
\begin{lstlisting}[language=bash,literate={-}{-}1,keywords={},upquote=true]
# Add OpenHPC repo 
[sms](*\#*) (*\chrootinstall*) http://repos.openhpc.community/OpenHPC/3/EL_9/x86_64/ohpc-release-3-1.el9.x86_64.rpm"
\end{lstlisting}
% end_ohpc_run

The {\em compute} nodes also need access to the EPEL repository, a required
dependency for \OHPC{} packages. 

% begin_ohpc_run
% ohpc_comment_header Configure access to EPEL repo
\begin{lstlisting}[language=bash,literate={-}{-}1,keywords={},upquote=true]
# Add epel repo
[sms](*\#*) nodeshell compute "dnf -y  install epel-release"

\end{lstlisting}
% end_ohpc_run


\noindent Additionally, a workaround is needed for \OHPC{} documentation files,
which are installed into a read-only NFS share /opt/ohpc/pub. Any package
attempting to write to that directory will fail to install. The following
prevents that by directing \texttt{rpm} not to install documentation files on
the {\em compute} nodes:

% begin_ohpc_run
\begin{lstlisting}[language=bash,literate={-}{-}1,keywords={},upquote=true]
[sms](*\#*) nodeshell compute echo -e %_excludedocs 1 \>\> ~/.rpmmacros
\end{lstlisting}
% end_ohpc_run

\noindent Now \OHPC{} and other cluster-related software components can be
installed on the nodes. The first step is to install a base compute package:
% begin_ohpc_run
% ohpc_comment_header Add OpenHPC base components to compute image
\begin{lstlisting}[language=bash,literate={-}{-}1,keywords={},upquote=true]
# Install compute node base meta-package
[sms](*\#*) (*\chrootinstall*) ohpc-base-compute
\end{lstlisting}
% end_ohpc_run

\noindent Next, we can include additional components:


%\newpage
% begin_ohpc_run
% ohpc_validation_comment Add OpenHPC components to compute instance
\begin{lstlisting}[language=bash,literate={-}{-}1,keywords={},upquote=true]
# Add Slurm client support meta-package
[sms](*\#*) (*\chrootinstall*) ohpc-slurm-client

# Add Network Time Protocol (NTP) support
[sms](*\#*) (*\chrootinstall*) ntp

# Add kernel drivers
[sms](*\#*) (*\chrootinstall*) kernel

# Include modules user environment
[sms](*\#*) (*\chrootinstall*)  --enablerepo=powertools lmod-ohpc
\end{lstlisting}
% end_ohpc_run

% ohpc_comment_header Optionally add InfiniBand support services in compute node image \ref{sec:add_components}
% ohpc_command if [[ ${enable_ib} -eq 1 ]];then
% ohpc_indent 5
\begin{lstlisting}[language=bash,literate={-}{-}1,keywords={},upquote=true]
# Optionally add IB support and enable
[sms](*\#*) (*\groupchrootinstall*) "InfiniBand Support"
\end{lstlisting}
% ohpc_indent 0
% ohpc_command fi
% end_ohpc_run

\vspace*{-0.25cm}
\subsubsection{Customize system configuration} \label{sec:master_customization}
Here we set up \NFS{}  mounting of a
\$HOME file system and the public \OHPC{} install path (\texttt{/opt/ohpc/pub})
that will be hosted by the {\em master} host in this  example configuration.

\vspace*{0.15cm}
% begin_ohpc_run
% ohpc_comment_header Customize system configuration \ref{sec:master_customization}
\begin{lstlisting}[language=bash,literate={-}{-}1,keywords={},upquote=true]
# Disable /tftpboot and /install export entries
[sms](*\#*) perl -pi -e "s|/tftpboot|#/tftpboot|" /etc/exports
[sms](*\#*) perl -pi -e "s|/install|#/install|" /etc/exports

# Export /home and OpenHPC public packages from master server
[sms](*\#*) echo "/home *(rw,no_subtree_check,fsid=10,no_root_squash)" >> /etc/exports
[sms](*\#*) echo "/opt/ohpc/pub *(ro,no_subtree_check,fsid=11)" >> /etc/exports
[sms](*\#*) exportfs -a
[sms](*\#*) systemctl restart nfs-server
[sms](*\#*) systemctl enable nfs-server

# Create NFS client mounts of /home and /opt/ohpc/pub on compute hosts
[sms](*\#*) nodeshell compute echo \
        "\""${sms_ip}:/home /home nfs nfsvers=3,nodev,nosuid 0 0"\"" \>\> /etc/fstab
[sms](*\#*) nodeshell compute echo \
        "\""${sms_ip}:/opt/ohpc/pub /opt/ohpc/pub nfs nfsvers=3,nodev 0 0"\"" \>\> /etc/fstab
[sms](*\#*) nodeshell compute systemctl restart nfs

# Mount NFS shares
[sms](*\#*) nodeshell compute mount /home
[sms](*\#*) nodeshell compute mkdir -p /opt/ohpc/pub
[sms](*\#*) nodeshell compute mount /opt/ohpc/pub

\end{lstlisting}
% end_ohpc_run



% Additional commands when additional computes are requested

% begin_ohpc_run
% ohpc_validation_newline
% ohpc_validation_comment Update basic slurm configuration if additional computes defined
% ohpc_validation_comment This is performed on the SMS, nodes will pick it up config file is copied there later
% ohpc_command if [ ${num_computes} -gt 4 ];then
% ohpc_command    perl -pi -e "s/^NodeName=(\S+)/NodeName=${compute_prefix}[1-${num_computes}]/" /etc/slurm/slurm.conf
% ohpc_command    perl -pi -e "s/^PartitionName=normal Nodes=(\S+)/PartitionName=normal Nodes=${compute_prefix}[1-${num_computes}]/" /etc/slurm/slurm.conf
% ohpc_command fi
% end_ohpc_run

%\clearpage
\subsubsection{Additional Customization ({\em optional})} \label{sec:addl_customizations}
\input{common/compute_customizations_intro}

\paragraph{Increase locked memory limits}
\input{common/memlimits_stateful}

\paragraph{Enable ssh control via resource manager}
\input{common/slurm_pam_stateful}

\paragraph{Add \Lustre{} client} \label{sec:lustre_client}
\input{common/lustre-client}
\input{common/lustre-client-centos-stateful}
\input{common/lustre-client-post-stateful}

\vspace*{0.4cm}

\paragraph{Add \clustershell{}}
\input{common/clustershell}

\paragraph{Add \genders{}}
\input{common/genders}

\paragraph{Add Magpie}
\input{common/magpie}

\paragraph{Add \conman{}} \label{sec:add_conman}
\input{common/conman}

\paragraph{Add \nhc{}} \label{sec:add_nhc}
\input{common/nhc}
\input{common/nhc_slurm}

%\subsubsection{Identify files for synchronization} \label{sec:file_import}
%The \Confluent{} system includes functionality to synchronize files located on the
SMS server for distribution to managed hosts. This is one way to
distribute user credentials to {\em compute} nodes (alternatively, you may
prefer to use a central authentication service like LDAP). To import local file-based
credentials, issue the following to enable the {\em synclist} feature and
register user credential files:

% begin_ohpc_run
% ohpc_comment_header Import files \ref{sec:file_import}
\begin{lstlisting}[language=bash,literate={-}{-}1,keywords={},upquote=true,literate={BOSSHORT}{\baseosshort{}}1]
# Add desired credential files to synclist
[sms](*\#*) echo "/etc/passwd -> /etc/passwd" > /var/lib/confluent/public/os/rocky-9.4-x86_64-default
[sms](*\#*) echo "/etc/group -> /etc/group" >> /var/lib/confluent/public/os/rocky-9.4-x86_64-default
[sms](*\#*) echo "/etc/shadow -> /etc/shadow" >> /var/lib/confluent/public/os/rocky-9.4-x86_64-default
\end{lstlisting}
% \end_ohpc_run
%\noindent Similarly, to import the
%global Slurm configuration file and the
cryptographic
key
%and associated file permissions
that is required by the {\em munge}
authentication library to be available on every host in the resource management
pool, issue the following:

% begin_ohpc_run
\begin{lstlisting}[language=bash,literate={-}{-}1,keywords={},upquote=true]
[sms](*\#*) echo "/etc/munge/munge.key -> /etc/munge/munge.key" >> /var/lib/confluent/public/os/rocky-9.4-x86_64-default
\end{lstlisting}
% \end_ohpc_run

\begin{center}
\begin{tcolorbox}[]
\small
The ``\texttt{nodeapply compute -F}'' command can be used to distribute changes made to any
defined synchronization files on the SMS host. Users wishing to automate this process may
want to consider adding a crontab entry to perform this action at defined intervals.
\end{tcolorbox}
\end{center}

%%%\subsubsection{Optional kernel arguments} \label{sec:optional_kargs}
%%%\input{common/conman_post}

\section{Install \OHPC{} Development Components}
\input{common/dev_intro}

%\vspace*{-0.15cm}
%\clearpage
\subsection{Development Tools} \label{sec:install_dev_tools}
\input{common/dev_tools}

\vspace*{-0.15cm}
\subsection{Compilers} \label{sec:install_compilers}
\OHPC{} presently packages the \GNU{} compiler toolchain integrated with the
underlying Lmod modules system in a hierarchical fashion. The modules
system will conditionally present compiler-dependent software based on the
toolchain currently loaded.

% begin_ohpc_run
% ohpc_comment_header Install Compilers \ref{sec:install_compilers}
\begin{lstlisting}[language=bash]
[sms](*\#*) (*\install*) gnu14-compilers-ohpc
\end{lstlisting}
% end_ohpc_run

%%% The llvm compiler toolchains are also provided as a standalone additional
%%% compiler family (ie. users can easily switch between gcc/clang environments),
%%% but we do not provide the full complement of downstream library builds.
%%%
%%% % begin_ohpc_run
%%% % ohpc_comment_header Install llvm Compilers
%%% \begin{lstlisting}[language=bash]
%%% [sms](*\#*) (*\install*) llvm5-compilers-ohpc
%%% \end{lstlisting}
%%% % end_ohpc_run


%\clearpage
\subsection{MPI Stacks} \label{sec:mpi}
For MPI development and runtime support, \OHPC{} provides pre-packaged builds
for a variety of MPI families and transport layers. Currently available options
and their applicability to various network transports are summarized in
Table~\ref{table:mpi}.  The command that follows installs a starting set of MPI
families that are compatible with both ethernet and high-speed fabrics.

\iftoggleverb{isx86}
% x86_64

\begin{table}[h]
\caption{Available MPI variants} \label{table:mpi}
\centering
\begin{tabular}{@{\hspace*{0.2cm}} *5l @{}}    \toprule
                                  & Ethernet (TCP)                 & \InfiniBand{}                  & \IntelR{} Omni-Path            \\ \midrule
\rowcolor{black!10} MPICH (ofi) & \multicolumn{1}{c}{\checkmark} & \multicolumn{1}{c}{\checkmark} & \multicolumn{1}{c}{\checkmark} \\
 MPICH (ucx)       & \multicolumn{1}{c}{\checkmark} & \multicolumn{1}{c}{\checkmark} & \multicolumn{1}{c}{\checkmark} \\
\rowcolor{black!10} MVAPICH2                          &                                & \multicolumn{1}{c}{\checkmark} &                                \\
MVAPICH2 (psm2) &                              &                                & \multicolumn{1}{c}{\checkmark} \\
\rowcolor{black!10} OpenMPI (ofi/ucx)            & \multicolumn{1}{c}{\checkmark} & \multicolumn{1}{c}{\checkmark} & \multicolumn{1}{c}{\checkmark} \\
%\rowcolor{black!10} OpenMPI (PMIx) & \multicolumn{1}{c}{\checkmark} & \multicolumn{1}{c}{\checkmark} & \multicolumn{1}{c}{\checkmark} \\ \bottomrule
\end{tabular}
\end{table}

\else
% aarch64

\begin{table}[h]
\caption{Available MPI builds} \label{table:mpi}
\centering
\begin{tabular}{@{\hspace*{0.2cm}} *5l @{}}    \toprule
                                  & Ethernet (TCP)                 & \InfiniBand{}                              \\ \midrule
\rowcolor{black!10} MPICH         & \multicolumn{1}{c}{\checkmark} &                                            \\
\rowcolor{black!10} OpenMPI                           & \multicolumn{1}{c}{\checkmark} & \multicolumn{1}{c}{\checkmark} \\
\end{tabular}
\end{table}

\fi

% begin_ohpc_run
% ohpc_comment_header Install MPI Stacks \ref{sec:mpi}
% ohpc_command if [[ ${enable_mpi_defaults} -eq 1 ]];then
% ohpc_indent 5
\begin{lstlisting}[language=bash]
[sms](*\#*) (*\install*) openmpi5-pmix-gnu14-ohpc mpich-ofi-gnu14-ohpc
\end{lstlisting}
% ohpc_indent 0
% ohpc_command fi
% end_ohpc_run

Note that OpenHPC 2.x introduces the use of two related transport layers for
the MPICH and OpenMPI builds that support a variety of underlying
fabrics: \href{https://www.openucx.org}{UCX} (Unified Communication X)
and \href{https://ofiwg.github.io/libfabric/}{OFI} (OpenFabrics interfaces).
In the case of OpenMPI, a monolithic build is provided which supports both
transports and end-users can customize their runtime preferences with
environment variables. For MPICH, two separate builds are provided and the
example above highlighted installing the {\texttt ofi} variant.  However, the
packaging is designed such that both versions can be installed simultaneously
and users can switch between the two via normal module command
semantics. Alternatively, a site can choose to install the {\texttt ucx} variant
instead as a drop-in MPICH replacement:

\begin{lstlisting}[language=bash]
[sms](*\#*) (*\install*) mpich-ucx-gnu14-ohpc
\end{lstlisting}

In the case where both MPICH variants are installed, two modules will be
visible in the end-user environment and an example of this configuration is
highlighted is below.

\begin{lstlisting}[language=bash]
[sms](*\#*) module avail mpich

-------------------- /opt/ohpc/pub/moduledeps/gnu14---------------------
   mpich/3.4.3-ofi    mpich/3.4.3-ucx (D)
\end{lstlisting}

If your system includes \InfiniBand{} and you enabled underlying support in
\S\ref{sec:add_ofed} and \S\ref{sec:addl_customizations}, an additional
MVAPICH2 family is available for use:

% begin_ohpc_run
% ohpc_validation_newline
% ohpc_command if [[ ${enable_ib} -eq 1 ]];then
% ohpc_indent 5
\begin{lstlisting}[language=bash]
[sms](*\#*) (*\install*) mvapich2-gnu14-ohpc
\end{lstlisting}
% ohpc_indent 0
% ohpc_command fi
% end_ohpc_run

Alternatively, if your system includes \IntelR{} \OmniPath{}, use the (\texttt{psm2})
variant of MVAPICH2 instead:

% begin_ohpc_run
% ohpc_command if [[ ${enable_opa} -eq 1 ]];then
% ohpc_indent 5
\begin{lstlisting}[language=bash]
[sms](*\#*) (*\install*) mvapich2-psm2-gnu14-ohpc
\end{lstlisting}
% ohpc_indent 0
% ohpc_command fi
% end_ohpc_run

%%--
%% https://github.com/openhpc/ohpc/issues/1273
%% disabling until we can get pmix/openmpi/slurm to play nicely
%%--
%% An additional OpenMPI build variant is listed in Table~\ref{table:mpi} which
%% enables \href{https://pmix.github.io/pmix/}{\color{blue}{PMIx}} job launch
%% support for use with \SLURM{}. This optional variant is
%% available as \texttt{openmpi5-pmix-slurm-gnu9-ohpc}.


\subsection{Performance Tools} \label{sec:install_perf_tools}
\OHPC{} provides a variety of open-source tools to aid in application
performance analysis (refer to Appendix~\ref{appendix:manifest} for a listing
of available packages). This group of tools can be installed as follows:

% begin_ohpc_run
% ohpc_comment_header Install Performance Tools \ref{sec:install_perf_tools}
\begin{lstlisting}[language=bash,keywords={},literate={-}{-}1]
# Install perf-tools meta-package
[sms](*\#*) (*\install*) ohpc-gnu14-perf-tools
\end{lstlisting}
% end_ohpc_run


\subsection{Setup default development environment}
System users often find it convenient to have a default development environment
in place so that compilation can be performed directly for parallel programs
requiring MPI. This setup can be conveniently enabled via modules and the \OHPC{}
modules environment is pre-configured to load an \texttt{ohpc} module on login
(if present). The following package install provides a default
environment that enables autotools, the \GNU{} compiler toolchain, and the
OpenMPI stack.

% begin_ohpc_run
\begin{lstlisting}[language=bash]
[sms](*\#*) (*\install*) lmod-defaults-gnu14-openmpi5-ohpc
\end{lstlisting}
% end_ohpc_run

\begin{center}
\begin{tcolorbox}[]
\small
\iftoggleverb{isx86}
If you want to change the default environment from the suggestion above, \OHPC{}
also provides the \GNU{} compiler toolchain with the MPICH and MVAPICH2 stacks:
\fi

\iftoggleverb{isaarch}
If you want to change the default environment from the suggestion above, \OHPC{}
also provides additional default options using the \GNU{} compiler toolchain
with multiple MPICH variants or MVAPICH2. Relevant lmod-default packages names
are as follows:
\fi

\begin{itemize*}
\item lmod-defaults-gnu14-mpich-ofi-ohpc
\item lmod-defaults-gnu14-mpich-ucx-ohpc
\iftoggleverb{isx86}
\item lmod-defaults-gnu14-mvapich2-ohpc
\fi
\end{itemize*}
\end{tcolorbox}
\end{center}


%\vspace*{0.2cm}
\subsection{3rd Party Libraries and Tools} \label{sec:3rdparty}
\input{common/third_party_libs_intro}

\begin{center}
\begin{tcolorbox}[]
\small
\OHPC{}-provided 3rd party builds are configured to be installed
into a common top-level repository so that they can be easily exported to
desired hosts within the cluster. This common top-level path
(\path{/opt/ohpc/pub}) was previously configured to be mounted on {\em
 compute} nodes in \S\ref{sec:master_customization}, so the packages will be
immediately available for use on the cluster after installation on the {\em
 master} host.
\end{tcolorbox}
\end{center}

%\iftoggle{isCentOS}{\clearpage}
%\nottoggle{isCentOS}{\clearpage}

For convenience, \OHPC{} provides package aliases for these 3rd party libraries
and utilities that can be used to install available libraries for use with the
GNU compiler family toolchain. For parallel libraries, aliases are grouped by
MPI family toolchain so that administrators can choose a subset should they
favor a particular MPI stack.  Please refer to Appendix~\ref{appendix:manifest}
for a more detailed listing of all available packages in each of these functional
areas. To install all available package offerings within \OHPC{}, issue the
following:

% begin_ohpc_run
% ohpc_comment_header Install 3rd Party Libraries and Tools \ref{sec:3rdparty}
\begin{lstlisting}[language=bash,keywords={},upquote=true,keepspaces]
# Install 3rd party libraries/tools meta-packages built with GNU toolchain
[sms](*\#*) (*\install*) ohpc-gnu14-serial-libs
[sms](*\#*) (*\install*) ohpc-gnu14-io-libs
[sms](*\#*) (*\install*) ohpc-gnu14-python-libs
[sms](*\#*) (*\install*) ohpc-gnu14-runtimes
\end{lstlisting}
% end_ohpc_run





% begin_ohpc_run
% ohpc_command if [[ ${enable_mpi_defaults} -eq 1 ]];then
% ohpc_indent 5
\begin{lstlisting}[language=bash,keywords={},upquote=true,keepspaces]
# Install parallel lib meta-packages for all available MPI toolchains
[sms](*\#*) (*\install*) ohpc-gnu14-mpich-parallel-libs
[sms](*\#*) (*\install*) ohpc-gnu14-openmpi5-parallel-libs
\end{lstlisting}
% ohpc_indent 0
% ohpc_command fi
% ohpc_command if [[ ${enable_ib} -eq 1 ]];then
% ohpc_indent 5
% ohpc_command (*\install*) ohpc-gnu14-mvapich2-parallel-libs
% ohpc_indent 0
% ohpc_command fi
% ohpc_command if [[ ${enable_opa} -eq 1 ]];then
% ohpc_indent 5
% ohpc_command (*\install*) ohpc-gnu14-mvapich2-parallel-libs
% ohpc_indent 0
% ohpc_command fi
% end_ohpc_run


\vspace*{.6cm}
\subsection{Optional Development Tool Builds} \label{sec:3rdparty_intel}
\input{common/oneapi_enabled_builds_slurm}

\section{Resource Manager Startup} \label{sec:rms_startup}
\input{common/slurm_startup_stateful}

\section{Run a Test Job} \label{sec:test_job}
With the resource manager enabled for production usage, users should now be
able to run jobs. To demonstrate this, we will add a ``test'' user on the {\em master}
host that can be used to run an example job.

% begin_ohpc_run
\begin{lstlisting}[language=bash,keywords={}]
[sms](*\#*) useradd -m test
\end{lstlisting}
% end_ohpc_run

Next, the user's credentials need to be distributed across the cluster.
\Confluent{}'s \texttt{nodeappy} has a merge functionality that adds new entries into
credential files on {\em compute} nodes:

% begin_ohpc_run
\begin{lstlisting}[language=bash,keywords={}]
# Create a sync file for pushing user credentials to the nodes
[sms](*\#*) echo "/etc/passwd -> /etc/passwd" >> /var/lib/confluent/public/os/rocky-9.4-x86_64-default/syncfiles
[sms](*\#*) echo "/etc/group -> /etc/group"   >> /var/lib/confluent/public/os/rocky-9.4-x86_64-default/syncfiles
[sms](*\#*) echo "/etc/shadow -> /etc/shadow" >> /var/lib/confluent/public/os/rocky-9.4-x86_64-default/syncfiles
# Use Confluent to distribute credentials to nodes
[sms](*\#*) nodeapply -F compute 
\end{lstlisting}
% end_ohpc_run


~\\
\input{common/prun}

\iftoggle{isSLES_ww_slurm_x86}{\clearpage}
%\iftoggle{isxCAT}{\clearpage}

\subsection{Interactive execution}
To use the newly created ``test'' account to compile and execute the
application {\em interactively} through the resource manager, execute the
following (note the use of \texttt{prun} for parallel job launch which summarizes
the underlying native job launch mechanism being used):

\begin{lstlisting}[language=bash,keywords={}]
# Switch to "test" user
[sms](*\#*) su - test

# Compile MPI "hello world" example
[test@sms ~]$ mpicc -O3 /opt/ohpc/pub/examples/mpi/hello.c

# Submit interactive job request and use prun to launch executable
[test@sms ~]$ salloc -n 8 -N 2

[test@c1 ~]$ prun ./a.out

[prun] Master compute host = c1
[prun] Resource manager = slurm
[prun] Launch cmd = mpiexec.hydra -bootstrap slurm ./a.out

 Hello, world (8 procs total)
    --> Process #   0 of   8 is alive. -> c1
    --> Process #   4 of   8 is alive. -> c2
    --> Process #   1 of   8 is alive. -> c1
    --> Process #   5 of   8 is alive. -> c2
    --> Process #   2 of   8 is alive. -> c1
    --> Process #   6 of   8 is alive. -> c2
    --> Process #   3 of   8 is alive. -> c1
    --> Process #   7 of   8 is alive. -> c2
\end{lstlisting}

\begin{center}
\begin{tcolorbox}[]
The following table provides approximate command equivalences between SLURM and
OpenPBS:

\vspace*{0.15cm}
\input common/rms_equivalence_table
\end{tcolorbox}
\end{center}
\nottoggle{isCentOS}{\clearpage}

\iftoggle{isCentOS}{\clearpage}

\subsection{Batch execution}

For batch execution, \OHPC{} provides a simple job script for reference (also
housed in the \path{/opt/ohpc/pub/examples} directory. This example script can
be used as a starting point for submitting batch jobs to the resource manager
and the example below illustrates use of the script to submit a batch job for
execution using the same executable referenced in the previous interactive example.

\begin{lstlisting}[language=bash,keywords={}]
# Copy example job script
[test@sms ~]$ cp /opt/ohpc/pub/examples/slurm/job.mpi .

# Examine contents (and edit to set desired job sizing characteristics)
[test@sms ~]$ cat job.mpi
#!/bin/bash

#SBATCH -J test               # Job name
#SBATCH -o job.%j.out         # Name of stdout output file (%j expands to %jobId)
#SBATCH -N 2                  # Total number of nodes requested
#SBATCH -n 16                 # Total number of mpi tasks #requested
#SBATCH -t 01:30:00           # Run time (hh:mm:ss) - 1.5 hours

# Launch MPI-based executable

prun ./a.out

# Submit job for batch execution
[test@sms ~]$ sbatch job.mpi
Submitted batch job 339
\end{lstlisting}

\begin{center}
\begin{tcolorbox}[]
\small
The use of the \texttt{\%j} option in the example batch job script shown is a convenient
way to track application output on an individual job basis. The \texttt{\%j} token
is replaced with the \SLURM{} job allocation number once assigned (job~\#339 in
this example).
\end{tcolorbox}
\end{center}

\clearpage
\appendix
{\bf \LARGE \centerline{Appendices}} \vspace*{0.2cm}

\addcontentsline{toc}{section}{Appendices}
\renewcommand{\thesubsection}{\Alph{subsection}}

\input{common/automation_appendix}
\input{common/upgrade_stateful}
\input{common/test_suite}
\input{common/customization_appendix_centos}
../../warewulf/slurm/manifest.tex
\input{common/signature}


\end{document}

    

